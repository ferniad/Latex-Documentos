\documentclass[11pt, a4paper, spanish]{article}
\usepackage[spanish]{babel}
\usepackage[utf8]{inputenc}
\usepackage[T1]{fontenc}
\usepackage{geometry}
\geometry{a4paper, margin=2cm}

\title{\textbf{Reporte de Visita Técnica}}
\author{Daniel Fernando Aranda Contreras \\ daniel2221648@correo.uis.edu.co}
\date{\today}

\begin{document}

\maketitle

\section*{Información General}
\textbf{Fecha:} 16 de octubre de 2025 \\
\textbf{Lugar:} Laboratorio de Medidas Eléctricas - ESSA \\
\textbf{Propósito:} Conocer los procesos de inspección y calibración de equipos de medición eléctrica



\section*{Equipos y Patrones de Calibración}

El laboratorio emplea fuentes patrón de alta precisión para corriente alterna, con rangos típicos de 10 A\textsubscript{RMS} y 100 V\textsubscript{RMS}. Según los estándares de un laboratorio acreditado, se infiere que este equipo cumple con una baja incertidumbre de medición (probablemente < 0.1\%) y mantiene una trazabilidad metrológica certificada al Instituto Nacional de Metrología de Colombia (INM), asegurando la confiabilidad y validez legal de sus calibraciones. Las especificaciones exactas del equipo pueden confirmarse consultando la documentación técnica del laboratorio.\section*{Observaciones Técnicas}

Durante la visita se identificó que el control de humedad se realiza de manera reactiva: cuando los valores salen del rango permitido, se suspenden las mediciones hasta que las condiciones ambientales vuelven a los parámetros establecidos. Este método, aunque funcional, podría optimizarse con un sistema de control ambiental continuo. Otro aspecto susceptible de mejora es el robustecimiento de la trazabilidad y seguridad del proceso. Una medida recomendable sería reubicar el laboratorio en un área de acceso restringido y considerar la implementación de protocolos que salvaguarden la identidad del personal calibrador. Esto minimizaría cualquier riesgo potencial de influencia externa sobre los resultados, fortaleciendo la integridad y confiabilidad de las mediciones.

\section*{Aprendizaje Adquirido}

La demostración práctica de los procesos de inspección y calibración de equipos fue especialmente valiosa. Además, resultó muy instructivo conocer los diferentes métodos de alteración fraudulenta en medidores eléctricos y las implicaciones éticas que conllevan estas prácticas.

\section*{Conclusión}

La visita a los laboratorios de inspección y calibración de ESSA resultó muy enriquecedora. Si bien existe oportunidad de mejora en el control ambiental, los procedimientos técnicos implementados y la calidad de los trabajos realizados son adecuados. Agradecemos al personal técnico por su atención y disposición para compartir sus conocimientos.

\end{document}