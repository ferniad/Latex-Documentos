%%%%%%%%%%%%%%%%%%%%%%%%%%%%%%%%%%%%%%%%%%%%%%%%%%%%%%%%%%%%%%%%%%%%%%%%%%%%%%%%
% PLANTILLA PARA UNA EVALUACIÓN EN LATEX
% Autor: Gemini (Google AI)
% Fecha: 2024
%%%%%%%%%%%%%%%%%%%%%%%%%%%%%%%%%%%%%%%%%%%%%%%%%%%%%%%%%%%%%%%%%%%%%%%%%%%%%%%%

% --- PREÁMBULO ---
% Define el tipo de documento y el tamaño de la fuente. 'article' es estándar.
\documentclass[12pt]{article}

% --- PAQUETES ---
% Codificación de entrada para soportar caracteres en español (ñ, tildes)
\usepackage[utf8]{inputenc}

% Soporte para el idioma español (títulos, fechas, etc.)
\usepackage[spanish]{babel}

% Paquetes esenciales para matemáticas
\usepackage{amsmath}
\usepackage{amsfonts}
\usepackage{amssymb}

% Para incluir imágenes
\usepackage{graphicx}

% Para personalizar los márgenes de la página
% Opciones: a4paper, letterpaper. Margen de 2.5cm en todos los lados.
\usepackage[a4paper, margin=2.5cm]{geometry}

% Para personalizar encabezados y pies de página
\usepackage{fancyhdr}

% --- CONFIGURACIÓN DEL DOCUMENTO ---

% Configuración del encabezado y pie de página con 'fancyhdr'
\pagestyle{fancy}
\fancyhf{} % Limpia todos los campos del encabezado y pie de página


\rhead{} % Encabezado derecho (Nombre del profesor)

\cfoot{Página \thepage} % Pie de página central (Número de página)
\renewcommand{\headrulewidth}{0.4pt} % Línea decorativa bajo el encabezado
\renewcommand{\footrulewidth}{0.4pt} % Línea decorativa sobre el pie de página

% Comando para insertar la fecha actual
\newcommand{\fecha}{\today}


% --- INICIO DEL DOCUMENTO ---
\begin{document}

% --- BLOQUE DEL TÍTULO Y ENCABEZADO PRINCIPAL ---
\begin{center}
    {\normalsize Facultad de Ciencias Exactas}\\[0.5cm]
    {\Large\bfseries EXAMEN FINAL}\\[0.3cm]
    {\large Asignatura: Matemáticas}
\end{center}

\vspace{0.8cm} % Espacio vertical

% --- BLOQUE DE INFORMACIÓN DEL ESTUDIANTE ---
% \noindent evita la sangría del párrafo
\noindent
\begin{tabular}{p{0.7\textwidth} p{0.25\textwidth}}
    \textbf{Nombre del Estudiante:} \underline{\hspace{5.5cm}} & \textbf{Calificación:} \underline{\hspace{1.5cm}} \\[1cm] % Deja espacio para el nombre y la nota
    \textbf{Fecha:} \fecha & \\
\end{tabular}

\vspace{1cm}

% --- INSTRUCCIONES DEL EXAMEN ---
\begin{center}
    \textbf{Instrucciones Generales}
\end{center}
\begin{itemize}
    \item Lea cuidadosamente cada pregunta antes de responder.
    \item La duración del examen es de \textbf{1 hora y media}.
    \item No se permite el uso de calculadoras programables ni teléfonos móviles.
    \item Justifique claramente todos los procedimientos. Las respuestas sin justificación no serán válidas.
    \item Utilice únicamente tinta de color negro o azul.
\end{itemize}

\vspace{1cm}

\begin{enumerate}
    % --- PUNTO BONUS: NÚMEROS COMPLEJOS CON FRACCIONES ---
    \item \textbf{Punto Bonus.} (10 Puntos)
    
    Realice la siguiente operación con números complejos que tienen coeficientes fraccionarios. Exprese el resultado final en la forma estándar $a+bi$.
    \[
    \left( \frac{1}{2} + \frac{3}{4}i \right)\cdot 3 - \left( \frac{5}{3} - \frac{1}{6}i \right)\cdot 7
    \]



  % --- PREGUNTA 1: Potenciación y Radicación Compleja ---
    \item \textbf{Operaciones con Exponentes y Radicales.} Simplifique las siguientes expresiones hasta su mínima expresión. No se aceptan exponentes negativos o fraccionarios en el resultado final.
    
    \begin{itemize}
        \item[a)] (15 Puntos) Simplifique la siguiente expresión aplicando propiedades de la potenciación:
        \[
        \left( \frac{a^{-2/3} \cdot b^{3/2}}{a^{-1} \cdot b^{-1/3}} \right)^{-2} \cdot \sqrt[3]{a^2}
        \]
        
        \item[b)] (10 Puntos) Racionalice y simplifique completamente el siguiente cociente:
        \[
        \frac{3\sqrt{2} - \sqrt{5}}{2\sqrt{5} - \sqrt{2}}
        \]
    \end{itemize}
    
    \vspace{0.5cm}

    % --- PREGUNTA 2: Operaciones Combinadas con Números Complejos ---
    \item \textbf{Números Complejos.} Dados los números complejos $z_1 = 3 - 2i$ y $z_2 = -1 + 4i$. Calcule lo siguiente, expresando el resultado en la forma $a+bi$.
    
    \begin{itemize}
        \item[a)] (15 Puntos) El resultado de la operación:
        \[
        \frac{z_1 \cdot \overline{z_2}}{z_1 + z_2}
        \]
        (Donde $\overline{z_2}$ es el conjugado de $z_2$).
        
        \item[b)] (10 Puntos) Resuelva para $z$ en la siguiente ecuación:
        \[
        (2-i)z - (1+5i) = 4 - 3i
        \]
    \end{itemize}
    
    \vspace{0.5cm}

    % --- PREGUNTA 3: Geometría Analítica Fundamental ---
    \item \textbf{Análisis de Rectas.} Resuelva los siguientes problemas.
    
    \begin{itemize}
        \item[a)] (10 Puntos) Determine si las siguientes dos rectas son paralelas, perpendiculares o ninguna de las dos. Justifique su respuesta comparando sus pendientes ($m$).
        \[ L_1: y = -2x + 5 \quad \text{y} \quad L_2: 6x + 3y = -3 \]
        
        \item[b)] (15 Puntos) Encuentre la ecuación de la recta que pasa por el punto $A(1, -4)$ y es perpendicular a la recta que pasa por los puntos $B(-5, 2)$ y $C(3, 3)$. Exprese el resultado en la forma $y=mx+b$.
    \end{itemize}

    \vspace{0.5cm}
    
    % --- PREGUNTA 4: Sistema de Ecuaciones por Método Gráfico ---
    \item \textbf{Solución Gráfica de un Sistema de Ecuaciones.} (20 Puntos)
    
    Resuelva el siguiente sistema de ecuaciones lineales 2x2 utilizando \textbf{exclusivamente el método gráfico}.
    \[ 
    \begin{cases} 
    2x + y = 4 \\
    x - y = -1
    \end{cases} 
    \]
    
    \begin{itemize}
        \item Dibuje ambas rectas en un mismo plano cartesiano.
        \item Señale claramente el punto de intersección en la gráfica.
        \item Escriba la coordenada $(x, y)$ que representa la solución del sistema.
    \end{itemize}
    
    % --- PREGUNTA 5: Aplicación de Sistemas de Ecuaciones ---
    \item \textbf{Problema de Aplicación.} (10 Puntos) En una granja hay conejos y gallinas. Si se cuentan un total de 40 cabezas y 110 patas, plantee un sistema de ecuaciones lineales 2x2 que modele la situación y resuélvalo por el método de eliminación o sustitución para determinar cuántos conejos y cuántas gallinas hay.
    

    % --- PREGUNTA ADICIONAL OBLIGATORIA: Fracciones Algebraicas ---
    \item \textbf{Operación con Fracciones Algebraicas (Obligatorio).} (15 Puntos)
    
    Resuelva la siguiente suma de fracciones algebraicas. Para ello, factorice completamente los denominadores, encuentre el común denominador y simplifique el resultado final tanto como sea posible.
    \[
    \frac{x - 7}{x^2 - x - 6} + \frac{2}{x^2 + 5x + 6}
    \]
\end{enumerate}



% --- FIN DEL DOCUMENTO ---
\end{document}