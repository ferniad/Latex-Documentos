\section{Resumen}
El presente trabajo es basado en la propuesta de Vatia S.A. E.S.P. para implementar un sistema de gestión de la información energética mediante tecnologías (IOT). Este proyecto busca ofrecer una alternativa de bajo costo en monitoreo y control del sistema eléctrico, enfocado especialmente para usuarios con consumos medios y bajos. El estudio pretende analizar sus benéficos en términos de la eficiencia energética y su impacto en el mercado eléctrico colombiano.  


\section{Introducción}
El uso eficiente de la energía se ha convertido en un aspecto importante para garantizar la sostenibilidad y competitividad de los sistemas eléctricos. En Colombia, el aumento constante en la demanda de electricidad, impulsado por el crecimiento económico y el aumento en la población, exige la incorporación de herramientas que permitan a los usuarios conocer y gestionar su consumo. Sin embargo, se sabe que la mayoría de las soluciones disponibles en el mercado suelen tener costos elevados, lo que limita el acceso a los usuarios con consumos más bajos, precisamente aquellos que representan una gran parte del sistema eléctrico nacional.

Ante esta situación, las tecnologías asociadas al Internet de las Cosas (IoT) surgen como una alternativa para desarrollar sistemas de control y monitoreo más accesibles, debido a que estos recopilan, transmiten y procesan información eléctrica en tiempo real, lo cual facilita observar los consumos y datos en distintas interfaces. Al ser soluciones de bajo costo, resultan muy beneficiosas para impulsar la masificación de la gestión energética en sectores donde no se implementan este tipo de prácticas.


\section{Estructuración de selección del artículo}
Para la estructuración del artículo se definieron las siguientes etapas:

\begin{itemize}
    \item \textbf{Etapa 1:} Una vez seleccionado el artículo a trabajar, cada integrante del grupo inició la respectiva lectura y análisis del mismo.
    \item \textbf{Etapa 2:} Posteriormente, se realizó la distribución de tareas del informe entre los integrantes del grupo. La asignación quedó de la siguiente manera: Nicolás se encargó del resumen y la introducción; José David y Juan Andrés asumieron la elaboración de la sección técnica; finalmente, Daniel Fernando se encargó de las conclusiones y de la organización general del documento.
\end{itemize}

\section{Sección técnica}

\subsection{Objetivo}
El propósito general del proyecto es implementar un sistema de gestión y control remoto de la información energética que permita:
\begin{itemize}
    \item Medir y supervisar en tiempo real las diferentes variables eléctricas de los usuarios.
    \item Reunir toda la información en una plataforma accesible y confiable.
    \item Ofrecer herramientas de análisis para mejorar la eficiencia y el ahorro.
    \item Integrar servicios adicionales provenientes de otras soluciones con IoT.
\end{itemize}

\subsection{Materiales y métodos}
El proyecto siguió la metodología de innovación de Vatia S.A. E.S.P., la cual maneja un enfoque inspirado en \textit{Lean Startup}. Esta metodología permite validar una hipótesis con los clientes desde etapas tempranas, evitando a la empresa invertir grandes cantidades de dinero sin antes comprobar su viabilidad. Luego, se inició con la identificación de las necesidades y tendencias, que luego se plasmaron en un \textit{Canvas} de modelo de negocio y propuesta de valor.

Posteriormente, se implementó un producto mínimo viable de baja fidelidad con el fin de comprobar la idea con usuarios reales. Estos modelos tienen limitaciones técnicas, pero permitieron confirmar el interés del mercado.

Con base en esas pruebas, se desarrolló un producto mínimo viable de alta fidelidad, con todas las funcionalidades técnicas necesarias. Una vez superada la etapa de validación, el proyecto pasó a la fase de ejecución, en la cual se destinaron recursos, personal y metas de implementación.

\subsubsection{Desarrollo del producto mínimo viable de alta fidelidad}
Se fabricaron equipos de comunicación de bajo costo, compatibles con la normativa nacional, que autoriza la lectura de medidores de manera remota. Estos dispositivos se comunican mediante un software de tipo SCADA que reúne toda la información, permite proyectar consumos, generar alertas, emitir reportes automáticos y, por último, integrar servicios adicionales. Entre las funciones más relevantes se encuentran el monitoreo en tiempo real de variables eléctricas, la simulación de facturas, la verificación de las condiciones con las que la energía llega y la integración con equipos de distintas marcas a través de protocolos estándar como Modbus. Todo esto hace que el sistema sea escalable, replicable y adaptable a distintos perfiles de usuarios.

\subsubsection{Ejecución y salida al mercado}
Luego del desarrollo técnico del producto, se pasó a la fase de comercialización. Para esto, se implementaron materiales de apoyo como fichas técnicas, folletos y videos, y se conformó un equipo de ventas para impulsar el producto a nivel nacional. Los primeros clientes sirvieron como casos de referencia, lo que permitió mejorar los modelos de costos y precios, y demostrar el valor de la solución en la práctica.

\subsection{Fases desarrolladas previamente por Vatia S.A. E.S.P.}
Antes de este proyecto, Vatia S.A. E.S.P. ya había desarrollado prototipos que permitían el monitoreo básico de variables eléctricas. Ahora, se presentan los principales componentes del prototipo creado, donde se incluyen herramientas, interfaces de visualización e integración, que ya fueron probadas y validadas directamente con clientes reales.

\subsubsection{Sistema de medida y fuentes de datos adicionales}
Es el encargado de medir las variables eléctricas relacionadas con el consumo de energía y garantiza la calidad de la información que recibe el usuario a través de la Plataforma de gestión. Para ello, se emplean principalmente dos tipos de dispositivos: analizadores de red fijos y transformadores de corriente y voltaje. Estos elementos permiten obtener datos eléctricos en tiempo real y exportarlos mediante protocolos estándar.

\subsubsection{Herramienta de simulación de facturas}
La herramienta permite hacer seguimiento al consumo de energía mediante una factura simulada y flexible que se actualiza automáticamente. El usuario puede elegir el período de consulta y visualizar su consumo de forma similar a una factura real, con proyecciones de consumo y parámetros de calidad de energía.

\subsubsection{Herramienta de calidad de energía}
Esta herramienta proporciona toda la información del sistema para que usuarios con conocimiento eléctrico puedan diagnosticar la calidad del servicio; también permite visualizar gráficas de consumo de energía, tensiones fase-fase, corrientes por cada fase, potencia activa trifásica y factor de potencia trifásico, configurables en períodos de tiempo.

\subsubsection{Integración con equipos de diferentes marcas}
La plataforma integra equipos mediante el protocolo Modbus, lo que permite monitorear no solo variables eléctricas, sino también las que afectan la productividad del sistema. Para este diseño se tomaron en cuenta dos aspectos:
\begin{itemize}
    \item \textbf{Requerimientos de plataforma de lectura:} la conexión se establece vía TCP/IP, donde el sistema de medida actúa como servidor y la plataforma como cliente.
    \item \textbf{Requerimientos de comunicación:} según la ubicación del sistema de medida, se plantean tres escenarios:
    \begin{itemize}
        \item \textbf{Con cobertura celular:} se usa red móvil con SIM card y conexión serial (RS232/RS485).
        \item \textbf{Con cobertura celular o internet a menos de 1 km:} se emplean redes complementarias como radiofrecuencia para conectar equipos sin cobertura.
        \item \textbf{Con punto de red Ethernet:} el cliente habilita acceso a internet, usando un equipo con interfaz Ethernet y VPN para conexión segura con Vatia S.A. E.S.P.
    \end{itemize}
\end{itemize}

\subsubsection{Dispositivo celular}
El equipo se pensó para ser económico y confiable, puede enlazarse con los sistemas de medida mediante una conexión serial y, además, permite hacer revisiones a distancia sin necesidad de enviar personal. Solo requiere energía, señal móvil y una tarjeta SIM, mientras un regulador interno mantiene el buen funcionamiento de sus partes.

\subsubsection{Costo–Valor del dispositivo celular}
Los costos de cada parte del dispositivo de comunicación celular se estimaron a partir de las cotizaciones recolectadas durante el desarrollo del proyecto, tomando en cuenta su valor y relevancia en el diseño.

\subsubsection{Dispositivo para la red complementaria}
El dispositivo de red complementaria amplía la cobertura en zonas sin señal, como sótanos, y cumple requisitos homologables por la CRC. Requiere energía, datos de configuración y la información a transmitir. Sus funciones principales son procesar datos, habilitar la radiofrecuencia y la conexión serial. Los costos de sus componentes se calcularon con base en cotizaciones reales y su importancia en el diseño.

\subsubsection{Dispositivo para red Ethernet}
El diseño prioriza mantener un bajo costo, seguido de la conexión serial con los sistemas de medida y el cumplimiento normativo. Además, la configuración debe ser sencilla para los técnicos instaladores. Según el diagrama, la mayoría de los componentes se acercan a sus valores ideales; destaca el módulo Ethernet, con un costo y un valor relativo altos, ambos cercanos al 60\%.

\section{Conclusiones}
El proyecto desarrollado por Vatia S.A. E.S.P. es un referente para el avance en el campo del IoT en Colombia, gracias a su bajo costo y a su capacidad de integración en distintos ámbitos, más allá del uso doméstico de la energía, y orientado a usuarios de bajo y medio consumo. Además, si el uso del sistema se masifica, un complemento fundamental serían los sistemas domóticos (para uso residencial) o la inmótica (para un uso empresarial), lo que representaría un paso adelante: del monitoreo a la acción, transformando los datos en medidas de ahorro concretas. 