\section{Resumen}
El acceso a herramientas de gestión energética accesibles sigue siendo limitado en Colombia debido a los altos costos de las soluciones tradicionales, lo que afecta especialmente a los usuarios de consumo medio y bajo. En este contexto, las tecnologías de Internet de las Cosas (IoT) surgen como una opción innovadora y de bajo costo que permite recopilar, transmitir y procesar información en tiempo real, facilitando un mayor control sobre el uso de la energía.
Este trabajo tiene como propósito analizar la factibilidad y los beneficios de un sistema de gestión de información energética basado en IoT, tomando como referencia la propuesta de Vatia S.A. E.S.P. El estudio busca demostrar cómo la implementación de estas tecnologías puede facilitar la eficiencia energética y ampliar el acceso a prácticas de gestión en sectores donde usualmente no se aplican. Al ofrecer alternativas de bajo costo y fácil implementación, los sistemas IoT no solo promueven un uso más racional de la energía, sino que también contribuyen a mejorar la competitividad del mercado eléctrico colombiano y a fortalecer la sostenibilidad y modernización del sistema energético nacional.

\section{Introducción}
El sostenido crecimiento en la demanda eléctrica de Colombia, impulsado por desarrollo económico y demográfico plantea un desafío crítico para la sostenibilidad y competitividad del sistema energético nacional, lo cual exige la incorporación de herramientas que permitan a los usuarios conocer y gestionar su consumo, pero se sabe que la mayoría de las soluciones disponibles en el mercado suelen tener costos elevados, lo que limita el acceso a los usuarios con consumos más bajos, precisamente aquellos que representan una gran parte del sistema eléctrico nacional.  
\newline
Ante esta situación, las tecnologías asociadas al internet de las cosas (IOT) surgen como una alternativa para desarrollar sistemas de control y monitoreo más accesibles, debido a que estos recopilan, transmiten y procesan información eléctrica en tiempo real, lo cual facilita observar los consumos y datos en distintas interfaces. Al ser soluciones de bajo costo resultan muy beneficiosas para impulsar la masificación de la gestión energética en sectores donde no se implementan estos tipos de prácticas.  
\newline
Este trabajo tiene como propósito analizar la factibilidad y los beneficios de un sistema de gestión de información energética basado en tecnologías IoT, tomando como caso la propuesta de Vatia S.A. E.S.P. El estudio se enfoca en usuarios de consumo medio y bajo, con el fin de evaluar cómo estas soluciones pueden facilitar la eficiencia energética, mejorar la competitividad del mercado eléctrico colombiano y facilitar el acceso a herramientas de gestión.

Ante esta situación, las tecnologías asociadas al Internet de las Cosas (IoT) surgen como una alternativa para desarrollar sistemas de control y monitoreo más accesibles, debido a que estos recopilan, transmiten y procesan información eléctrica en tiempo real, lo cual facilita la observación de los consumos y datos en distintas interfaces. Al ser soluciones de bajo costo, resultan muy beneficiosas para impulsar la masificación de la gestión energética en sectores donde no se implementan este tipo de prácticas.

Este trabajo tiene como propósito analizar la factibilidad y los beneficios de un sistema de gestión de información energética basado en tecnologías IoT, tomando como caso la propuesta de Vatia S.A. E.S.P. \cite{Jaramillo2022}. El estudio se enfoca en usuarios de consumo medio y bajo, con el fin de evaluar cómo estas soluciones pueden facilitar la eficiencia energética, mejorar la competitividad del mercado eléctrico colombiano y facilitar el acceso a herramientas de gestión.

\section{Estructuración y selección del artículo}

Para la estructuración del artículo se definieron las siguientes etapas:

\begin{enumerate}
    \item Una vez seleccionado el artículo a trabajar \cite{Jaramillo2022}, cada integrante del grupo inició la respectiva lectura y análisis del mismo.
    \item Posteriormente, se realizó la distribución de tareas del informe entre los integrantes del grupo. La asignación quedó de la siguiente manera: Nicolás se encargó del resumen y la introducción; José David y Juan Andrés asumieron la elaboración de la sección técnica; finalmente, Daniel Fernando se encargó de las conclusiones y de la organización general del documento.
\end{enumerate}

\section{Sección técnica}

\subsection{Objetivo}
El propósito general del proyecto descrito en \cite{Jaramillo2022} es implementar un sistema de gestión y control remoto de la información energética que permita:
\begin{itemize}
    \item Medir y supervisar en tiempo real las diferentes variables eléctricas de los usuarios.
    \item Reunir toda la información en una plataforma accesible y confiable.
    \item Ofrecer herramientas de análisis para mejorar la eficiencia y el ahorro.
    \item Integrar servicios adicionales provenientes de otras soluciones con IoT.
\end{itemize}

\subsection{Materiales y métodos}
El proyecto siguió la metodología de innovación de Vatia S.A. E.S.P., la cual maneja un enfoque inspirado en Lean Startup \cite{Jaramillo2022}. Esta metodología permite validar una hipótesis con los clientes desde etapas tempranas, evitando a la empresa invertir grandes cantidades de dinero sin antes comprobar su viabilidad. Luego, se inició con la identificación de las necesidades y tendencias, que después se plasmaron en un Canvas de modelo de negocio y propuesta de valor.

Posteriormente, se implementó un producto mínimo viable de baja fidelidad, con el fin de comprobar la idea con usuarios reales. Estos modelos tienen limitaciones técnicas, pero permitieron confirmar el interés del mercado.

Con base en esas pruebas, se desarrolló un producto mínimo viable de alta fidelidad, con todas las funcionalidades técnicas necesarias. Una vez superada la etapa de validación, el proyecto pasó a la fase de ejecución, en la cual se destinaron recursos, personal y metas de implementación.

\subsection{Desarrollo del producto mínimo viable de alta fidelidad}
Se fabricaron equipos de comunicación de bajo costo, compatibles con la normativa nacional, que autoriza la lectura de medidores de manera remota \cite{Jaramillo2022}. Estos dispositivos se comunican mediante un software de tipo SCADA que reúne toda la información, permite proyectar consumos, generar alertas, emitir reportes automáticos y, por último, integrar servicios adicionales. Entre las funciones más relevantes se encuentran el monitoreo en tiempo real de variables eléctricas, la simulación de facturas, la verificación de las condiciones con las que la energía llega y la integración con equipos de distintas marcas a través de protocolos estándar como Modbus. Todo esto hace que el sistema sea escalable, replicable y adaptable a distintos perfiles de usuarios.

\subsection{Ejecución y salida al mercado}
Luego del desarrollo técnico del producto, se pasó a la fase de comercialización. Para esto, se implementaron materiales de apoyo como fichas técnicas, folletos y videos, y se conformó un equipo de ventas para impulsar el producto a nivel nacional. Los primeros clientes sirvieron como casos de referencia, lo que permitió mejorar los modelos de costos y precios y demostrar el valor de la solución en la práctica \cite{Jaramillo2022}.

\subsection{Fases desarrolladas previamente por Vatia S.A. E.S.P.}
Antes de este proyecto, Vatia S.A. E.S.P. ya había desarrollado prototipos que permitían el monitoreo básico de variables eléctricas. Ahora, se presentan los principales componentes del prototipo creado \cite{Jaramillo2022}, donde se incluyen herramientas, interfaces de visualización e integración, que ya fueron probadas y validadas directamente con clientes reales.

\subsubsection{Sistema de medida y fuentes de datos adicionales}
Es el encargado de medir las variables eléctricas relacionadas con el consumo de energía y garantiza la calidad de la información que recibe el usuario a través de la Plataforma de Gestión. Para ello, se emplean principalmente dos tipos de dispositivos: analizadores de red fijos y transformadores de corriente y voltaje. Estos elementos permiten obtener datos eléctricos en tiempo real y exportarlos mediante protocolos estándar.

\subsubsection{Herramienta de simulación de facturas}
La herramienta permite hacer seguimiento al consumo de energía mediante una factura simulada y flexible que se actualiza automáticamente. El usuario puede elegir el periodo de consulta y visualizar su consumo de forma similar a una factura real, con proyecciones de consumo y parámetros de calidad de energía.

\subsubsection{Herramienta de calidad de energía}
Esta herramienta proporciona toda la información del sistema para que usuarios con conocimiento eléctrico puedan diagnosticar la calidad del servicio. También permite visualizar gráficas de consumo de energía, tensiones fase-fase, corrientes por cada fase, potencia activa trifásica y factor de potencia trifásico, configurables en periodos de tiempo.

\subsubsection{Integración con equipos de diferentes marcas}
La plataforma integra equipos mediante el protocolo Modbus, lo que permite monitorear no solo variables eléctricas sino también las que afectan la productividad del sistema \cite{Jaramillo2022}. Para este diseño se tomaron en cuenta dos aspectos:
\begin{itemize}
    \item \textbf{Requerimientos de plataforma de lectura:} la conexión se establece vía TCP/IP, donde el sistema de medida actúa como servidor y la plataforma como cliente.
    \item \textbf{Requerimientos de comunicación:} según la ubicación del sistema de medida, se plantean tres escenarios:
    \begin{itemize}
        \item \textit{Con cobertura celular:} se usa red móvil con SIM card y conexión serial (RS232/RS485).
        \item \textit{Con cobertura celular o internet a menos de 1 km:} se emplean redes complementarias como radiofrecuencia para conectar equipos sin cobertura.
        \item \textit{Con punto de red Ethernet:} el cliente habilita acceso a internet, usando un equipo con interfaz Ethernet y VPN para una conexión segura con Vatia S.A. E.S.P.
    \end{itemize}
\end{itemize}

\subsection{Dispositivo celular}
El equipo se pensó para ser económico y confiable, puede enlazarse con los sistemas de medida mediante una conexión serial y, además, permite hacer revisiones a distancia sin necesidad de enviar personal. Solo requiere energía, señal móvil y una tarjeta SIM, mientras un regulador interno mantiene el buen funcionamiento de sus partes.

\subsubsection{Costo-Valor del dispositivo celular}
Los costos de cada parte del dispositivo de comunicación celular se estimaron a partir de las cotizaciones recolectadas durante el desarrollo del proyecto, tomando en cuenta su valor y relevancia en el diseño \cite{Jaramillo2022}.

\subsection{Dispositivo para la red complementaria}
El dispositivo de red complementaria amplía la cobertura en zonas sin señal, como sótanos, y cumple requisitos homologables por la CRC. Requiere energía, datos de configuración y la información a transmitir. Sus funciones principales son procesar datos, habilitar la radiofrecuencia y la conexión serial. Los costos de sus componentes se calcularon con base en cotizaciones reales y su importancia en el diseño.

\subsection{Dispositivo para red Ethernet}
% Ojo: Esta sección estaba vacía en el texto original.
El diseño prioriza mantener un bajo costo, seguido de la conexión serial con los sistemas de medida y el cumplimiento normativo. Además, la configuración debe ser sencilla para los técnicos instaladores. Según el diagrama, la mayoría de los componentes se acercan a sus valores ideales; destaca el módulo Ethernet, con un costo y un valor relativo altos, ambos cercanos al 60\,\%.

\section{Implementación}
Para llevar a cabo la implementación del sistema de monitoreo y control automático, se arrancó con una etapa de pruebas básicas en laboratorio. Ahí se verificó que los equipos de comunicación diseñados localmente funcionaran bien con los medidores de energía, revisando voltajes, estabilidad y que los datos viajaran sin problema \cite{Jaramillo2022}. Una vez superado ese filtro, se pasó a pruebas en campo con clientes en distintas ciudades del país, lo que permitió comprobar que el sistema operaba en condiciones reales usando redes celulares, Ethernet y también enlaces de radiofrecuencia.

Durante el proceso se usaron plataformas de gestión tipo SCADA, que facilitaron la integración de variables eléctricas con otra información de interés para cada usuario. Todo esto se hizo bajo la lógica de producto viable, es decir, se sacaban versiones sencillas, se probaban con usuarios y, a partir de su opinión, se ajustaba el diseño. Gracias a ese método, el sistema tomó forma, llegando así a una versión más completa y estable.

El sistema no solo permite medir y visualizar, sino que también ofrece herramientas como la proyección del gasto energético, simulación de facturas y reportes de calidad eléctrica. Con estas funciones, los clientes pueden identificar cuándo tomar decisiones para reducir sus costos energéticos.

Ya en la etapa comercial, se instalaron equipos en sectores como industrias y instituciones educativas, lo cual demostró que la solución es escalable y se adapta a distintas necesidades. Además, el uso de tecnologías IoT de bajo costo hizo posible que usuarios con consumos medios o bajos —quienes usualyacceden a este tipo de herramientas por lo costoso de los equipos importados— pudieran contar con una solución efectiva y ajustada a sus realidades.

En resumen, la implementación dejó claro que sí es posible combinar innovación, economía y cumplimiento normativo para entregar un sistema confiable, sostenible y capaz de integrar servicios adicionales en el futuro, fortaleciendo así la eficiencia energética y la competitividad de las empresas en el contexto colombiano \cite{Jaramillo2022}.

\section{Beneficios}
La principal ventaja de este sistema es que reduce la barrera de acceso a la eficiencia energética. Antes, solo las empresas grandes podían pagar equipos importados para medir y controlar el consumo; ahora, con la solución desarrollada localmente por Vatia \cite{Jaramillo2022}, los costos bajan y hasta negocios pequeños o instituciones con consumos moderados pueden usarla sin problema.

Otro beneficio clave es el ahorro económico. Como los dispositivos son locales, no hay que gastar en importaciones y, además, se ajustan mejor a las condiciones del país. Encima, la plataforma muestra en tiempo real el gasto y hasta simula facturas, lo que ayuda a gestionar los gastos y a evitar sorpresas en la factura.

También se obtiene una mejora en la organización y competitividad. Tener datos claros del consumo permite que las empresas tomen decisiones rápidas, como apagar lo que no se está usando, mejorar procesos o evitar pérdidas de energía. Eso, al final, se traduce en menos costos y más productividad.

El sistema no se queda solo en medir voltajes o corrientes. Se puede conectar con otras variables como temperatura o humedad, lo que lo hace útil para muchos sectores: agricultura, comercio, industria o cualquier negocio que dependa de varias condiciones al mismo tiempo. Y, por supuesto, el tema ambiental es muy importante. Al consumir la energía de manera más eficaz, se reduce el desperdicio y, con eso, también el impacto negativo en el medio ambiente. Esto es un beneficio doble porque se ahorra dinero y se protege el planeta.

\section{Análisis propio}
Este trabajo se centra en cómo se diseñó un sistema para monitorear y controlar el consumo de energía eléctrica usando tecnologías del Internet de las Cosas (IoT), según lo presentado por \cite{Jaramillo2022}. La idea principal es que los usuarios, sobre todo los que tienen consumos pequeños o medianos, puedan ver en tiempo real cuánto gastan, recibir alertas, proyectar sus facturas y todo desde una plataforma digital.

Lo que plantean los autores es muy interesante porque ataca un problema que vivimos en Colombia: se desperdicia una gran cantidad de energía, más del 50\,\% según el Ministerio de Minas. Y muchas veces no se aplican soluciones de eficiencia porque los equipos que permiten medir y controlar son muy costosos o no se adaptan a las condiciones del mercado local. Entonces, lo que hace Vatia S.A. E.S.P. con este proyecto es fabricar sus propios dispositivos a bajo costo, que además cumplen con las normas y se conectan fácilmente por redes celulares, Ethernet o radiofrecuencia, dependiendo de su ubicación.

Algo que resalto es que no se quedaron en el papel. Hicieron prototipos, los probaron en laboratorio y después con clientes reales: empresas, un zoológico y hasta una hidroeléctrica. Eso le da peso a la investigación porque no es solo teoría, sino que ya tiene aplicaciones prácticas. Incluso se nota que los equipos se comportaron excelente, incluso en condiciones difíciles, y demostraron buena confiabilidad en la transmisión de datos \cite{Jaramillo2022}.

También hay un punto clave: el sistema no se limita a medir tensiones y consumos, sino que puede integrar otras variables como temperatura y humedad. Eso abre el camino para que no solo sea útil en el sector eléctrico, sino también en procesos industriales, agrícolas o comerciales. En pocas palabras, es una solución adaptable.

Desde mi punto de vista, el aporte más grande de este proyecto está en democratizar el acceso a la eficiencia energética. Antes, este tipo de sistemas estaban reservados para grandes empresas; ahora, con un desarrollo local, cualquier persona o institución puede tener un control más claro sobre su gasto de energía y tomar decisiones para ahorrar.

\section{Conclusiones}
El proyecto desarrollado por Vatia S.A. E.S.P. \cite{Eltamaly2021},s es un referente para el avance en el campo del IoT en Colombia, gracias a su bajo costo y a su capacidad de integración en distintos ámbitos, más allá del uso doméstico de la energía, y orientado a usuarios de bajo y medio consumo. Además, si el uso del sistema se masifica, un complemento fundamental serían los sistemas domóticos (para uso residencial) o la inmótica (para un uso empresarial), lo que representaría un paso adelante: del monitoreo a la acción, transformando los datos en medidas de ahorro concretas.
