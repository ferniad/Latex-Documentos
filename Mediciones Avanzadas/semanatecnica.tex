\documentclass{article}
\usepackage[utf8]{inputenc}
\usepackage[spanish]{babel}
\usepackage{amsmath}

\title{Análisis de Charlas de la Semana Técnica}
\author{Daniel Fernando Aranda Contreras - 2221648}
\date{\today}

\begin{document}
\maketitle

\section*{Charla de Normatividad en Hospitales}
La charla del inspector sobre la normativa eléctrica en Colombia fue muy valiosa. Aunque el tema es técnico y legal, la presentación fue muy dinámica y para nada aburrida.

Lo que más me gustó fue cómo el inspector se mantuvo activo, involucrando a todos los estudiantes. No se sintió como un monólogo, sino como un diálogo, lo que hizo que fuera mucho más fácil entender todo. En resumen, fue una charla muy valiosa y superó mis expectativas.

\section*{Ciberseguridad en Redes Eléctricas Inteligentes}
En relación con la exposición del profesor César sobre las redes de ciudades inteligentes y redes eléctricas inteligentes, se destaca la \textbf{ciberseguridad} como un aspecto crucial que merece mayor atención. Dada la naturaleza centralizada de la arquitectura de estos sistemas, la implementación de protocolos de comunicación intrínsecamente robustos es fundamental.

La vulnerabilidad de un sistema de control centralizado representa un riesgo significativo. Un acceso no autorizado podría comprometer la integridad y disponibilidad del sistema eléctrico, lo que subraya que las medidas de ciberseguridad son una necesidad imperante para la protección de la infraestructura crítica.

\section*{Desafíos Sociales en la Creación de Comunidades Energéticas}
En el contexto del sector energético colombiano, se identifica como un desafío primordial la \textbf{cohesión social}. En diversas comunidades, se observan fenómenos de intolerancia y de una marcada competitividad individualista, lo que obstaculiza la colaboración y el trabajo en equipo.

Por lo tanto, se considera que una fase preliminar y crucial para cualquier iniciativa energética es la \textbf{creación y el fortalecimiento de la comunidad}. Este proceso es esencial en aquellos entornos donde no existen estructuras de cooperación, ya que la sinergia comunitaria es un pilar indispensable para la correcta implementación y sostenibilidad de proyectos.

\section*{Reflexión sobre la Adopción de la Inteligencia Artificial}
En las charlas que involucraban la Inteligencia Artificial (IA), se discutió que, si bien es una tecnología con considerable potencial, su adopción debe ser abordada con cautela, evitando la precipitación. Es fundamental no comprometerse de manera total e inmediata con las nuevas innovaciones tecnológicas.

Considero perjudicial la priorización \textbf{exclusiva} de las disciplinas \textbf{STEM} (Ciencia, Tecnología, Ingeniería y Matemáticas) en detrimento de un enfoque más holístico. Es indispensable fortalecer la perspectiva \textbf{social} en el desarrollo e implementación de estas tecnologías. Un enfoque equilibrado, que integre las humanidades y las ciencias sociales, es crucial para asegurar que el progreso tecnológico beneficie a la sociedad en su conjunto, y no se limite únicamente a la innovación por sí misma.

\end{document}