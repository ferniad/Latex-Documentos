
\documentclass{beamer}

\usetheme{Madrid} % Puedes elegir otro tema si lo prefieres
\usepackage[utf8]{inputenc}
\usepackage{amsmath}
\usepackage{amssymb}
\usepackage{graphicx}
\usepackage{multicol}

\title{Análisis Termodinámico de un Ciclo Rankine Solar con Tolueno}
\author{Daniel Fernando Aranda Contreras, Jeremy Carreño Fontalvo, Santiago Silva Quintero}
\institute{Escuela E3T, Universidad Industrial de Santander}
\date{\today}

\begin{document}

\frame{\titlepage}

\begin{frame}
    \frametitle{Introducción y Conceptos Clave}
    \begin{itemize}
        \item \textbf{Título del Estudio:} ANÁLISIS TERMODINÁMICO DE UN CICLO RANKINE SOLAR CON TOLUENO COMO FLUIDO DE TRABAJO Y RECUPERACIÓN DE CALOR.
        \item \textbf{Autores:} Daniel Fernando Aranda Contreras, Jeremy Carreño Fontalvo, Santiago Silva Quintero (Escuela E3T, Universidad Industrial de Santander).
        \item \textbf{Resumen del Estudio:}
        \begin{itemize}
            \item Aborda el análisis termodinámico de una planta de energía solar basada en un ciclo Rankine.
            \item Utiliza la energía solar como fuente de calor primaria.
            \item Se emplea tolueno como fluido de trabajo, justificado por la baja temperatura operativa del ciclo en comparación con el agua.
        \end{itemize}
        \item \textbf{Componentes Principales del Sistema:}
        \begin{itemize}
            \item Calentador, recalentador, turbina de alta presión, turbina de baja presión, intercambiador de calor recuperativo, bomba y condensador.
        \end{itemize}
        \item \textbf{Objetivos del Estudio:}
        \begin{itemize}
            \item Analizar y calcular las presiones en cada estado, el trabajo específico de las turbinas y la bomba, y la transferencia de calor en los componentes.
            \item Determinar la eficiencia térmica global del ciclo y verificar el cumplimiento de la primera y segunda ley de la termodinámica.
            \item Presentar un diagrama T-s del ciclo e investigar la influencia de la presión de recalentamiento sobre la eficiencia.
            \item Identificar el equipo con mayor generación de entropía para identificar áreas de mejora.
        \end{itemize}
    \end{itemize}
\end{frame}

\begin{frame}
    \frametitle{Caso de Estudio y Descripción del Ciclo}
    \begin{itemize}
        \item Una planta de energía solar con un ciclo Rankine que utiliza energía solar como fuente de calor.
        \item Receptores parabólicos concentran la energía solar en una tubería con fluido de transferencia de calor.
        \item El fluido de transferencia de calor entra a la planta a $T_{f,in} = 288^{\circ}C$.
        \item \textbf{Fluido de Trabajo:} Se usa tolueno debido a la baja temperatura de trabajo del ciclo, lo que lo hace más eficiente que el agua.
        \item \textbf{Puntos Clave del Ciclo:}
        \begin{itemize}
            \item Tolueno sale del calentador a $T_1 = T_{f,in} - \Delta T_H$, con $\Delta T_H = 20~K$.
            \item Expansión en turbina de alta presión (HPt) de $P_1 = P_{high} = 1034~kPa$ a $P_2 = P_{reheat} = 250~kPa$, con $\eta_{HPt} = 0.81$.
            \item Recalentamiento a $T_3 = T_{f,in} - \Delta T_{RH}$, con $\Delta T_{RH} = 20~K$.
            \item Expansión en turbina de baja presión (LPt) con $\eta_{LPt} = 0.78$.
            \item Presión de condensación ajustada para tolueno como líquido saturado a $T_6 = T_{amb} + \Delta T_c$, donde $T_{amb} = 35^{\circ}C$ y $\Delta T_c = 15~K$.
            \item Bombeo a alta presión con eficiencia $\eta_p = 0.6$.
            \item Recuperación de calor del tolueno que sale de la turbina de baja presión mediante un intercambiador de calor regenerativo.
            \item Tolueno entra al condensador a $T_5 = T_7 - \Delta T_r$, donde $\Delta T_r = 20~K$.
            \item Se desprecia la caída de presión en calentador, recalentador, intercambiador recuperativo y condensador.
        \end{itemize}
    \end{itemize}
\end{frame}

\begin{frame}
    \frametitle{Cálculos y Resultados Clave}
    \begin{multicols}{2}
    \begin{itemize}
        \item \textbf{Presiones en los Estados:}
        \begin{itemize}
            \item $P_1 = 1034~kPa$ 
            \item $P_2 = 250~kPa$ 
            \item $P_3 = 250~kPa$ 
            \item $P_4 = P_5 = P_6 = 12.29~kPa$ (presión de condensación) 
            \item $P_6 = 12.13~kPa$ (para $T_6 = 323.15~K$ o $50^{\circ}C$) 
            \item $P_7 = P_6$ (salida de la bomba) 
            \item $P_8 = P_1 = 1034~kPa$ (entrada a la bomba) 
        \end{itemize}
        \item \textbf{Transferencias de Calor Específicas (kJ/kg):} 
        \begin{itemize}
            \item Calentador ($q_H$): $h_1 - h_7 = 727.394$ 
            \item Recalentador ($q_{RH}$): $h_3 - h_2 = 67.463$ 
            \item Condensador ($q_C$): $q_8 - q_5 = -639.679$  (Nota: $Q_{salida}=h_5-h_6=641.71$ )
            \item Recuperador ($q_r$): $h_2 - h_5 = 641.709$ 
        \end{itemize}
        \item \textbf{Trabajo Específico (kJ/kg):}
        \begin{itemize}
            \item Turbina de Alta Presión ($W_{HPt}$): $h_1 - h_2 = 49.697$ 
            \item Turbina de Baja Presión ($W_{LPt}$): $h_4 - h_3 = 105.483$  (Nota: En el documento, $W_{LPt}=h_3-h_4=105.483$. Hay una inconsistencia en el signo entre el texto y la tabla).
            \item Bomba ($W_P$): $h_7 - h_6 = 2.030$ 
        \end{itemize}
    \end{itemize}
    \columnbreak
    \begin{itemize}
        \item \textbf{Generación de Entropía Específica (kJ/kg$\cdot$K):}
        \begin{itemize}
            \item Turbina de Alta Presión ($s_{gen,HPt}$): $s_2 - s_1 = 0.231$ 
            \item Turbina de Baja Presión ($s_{gen,LPt}$): $s_4 - s_3 = 0.06$ 
            \item Bomba ($s_{gen,P}$): $s_7 - s_6 = 0.006$ 
            \item Recuperador ($s_{gen,r}$): $-1.837$  (Nota: Un valor negativo para la generación de entropía total es termodinámicamente incorrecto. Podría ser un error de signo o una referencia a una $\Delta s$ específica de un flujo).
        \end{itemize}
        \item \textbf{Eficiencia Térmica Global ($\eta$):} 
        \begin{itemize}
            \item $\eta = \frac{W_{neto}}{Q_{suministrado}} = \frac{W_{HPt} + W_{LPt} - W_P}{q_H + q_{RH}}$ 
            \item $\eta = \frac{49.697 + 105.483 - 2.030}{727.394 + 67.463} = 0.1927 \approx 19.27\%$ 
        \end{itemize}
        \item \textbf{Verificación de la Primera Ley de la Termodinámica:} 
        \begin{itemize}
            \item $W_{neto} = W_t + W_{t,lp} - W_b = 49.69 + 105.48 - 0 = 155.17~kJ/kg$  (Nota: $W_b=0$ en esta ecuación, pero se calculó $W_P=2.030~kJ/kg$ en el texto. Esto podría indicar una bomba ideal en el cálculo de $W_{neto}$ o una diferencia en la interpretación de los trabajos).
            \item $Q_{neto} = Q_{entra} - (h_4 - h_5) - Q_{salida} = 794.85 - 0 - 641.71 = 153.14~kJ/kg$ 
            \item Se corrobora que $W_{neto} \approx Q_{neto}$ con un pequeño margen de error.
        \end{itemize}
        \item \textbf{Verificación de la Segunda Ley de la Termodinámica:} 
        \begin{itemize}
            \item Para un ciclo cerrado ideal, $\Delta S_{ciclo}=0$.
            \item Los valores de $\Delta s$ para cada proceso se presentan en el Cuadro III del informe.
        \end{itemize}
    \end{itemize}
    \end{multicols}
\end{frame}

\begin{frame}
    \frametitle{Diagrama T-s y Conclusiones}
    \begin{center}
        % Diagrama T-s del Ciclo Rankine Solar (imagen no encontrada)
    \end{center}
    \begin{itemize}
        \item El diagrama T-s visualiza el comportamiento del ciclo y sus estados termodinámicos.
        \item \textbf{Conclusiones:} 
        \begin{itemize}
            \item Los datos termodinámicos (Cuadro I) son fundamentales para el análisis energético y exergético del ciclo.
            \item La coherencia de estos datos es crucial para verificar el cumplimiento de la primera y segunda ley de la termodinámica (Cuadros II y III).
            \item El uso de tolueno evidencia beneficios en sistemas de recuperación de calor a temperaturas intermedias.
            \item La identificación del equipo con mayor generación de entropía permite áreas de mejora en el diseño y operación del ciclo.
        \end{itemize}
    \end{itemize}
\end{frame}

\begin{frame}
    \frametitle{Referencias}
    \begin{thebibliography}{9}
        \bibitem[1]{cite1} Aranda Contreras, D. F., Carreño Fontalvo, J., \& Silva Quintero, S. (s.f.). ANÁLISIS TERMODINÁMICO DE UN CICLO RANKINE SOLAR CON TOLUENO COMO FLUIDO DE TRABAJO Y RECUPERACIÓN DE CALOR. \textit{Escuela E3T, Universidad Industrial de Santander}. 
        \bibitem[2]{cite2} Resumen. (s.f.). 
        \bibitem[3]{cite3} Se emplea tolueno como fluido de trabajo, una elección justificada por la baja temperatura operativa del ciclo en comparación con el agua. (s.f.). 
        \bibitem[4]{cite4} El sistema incorpora un calentador, un recalentador, una turbina de alta presión, una turbina de baja presión, un intercambiador de calor recuperativo, una bomba y un condensador. (s.f.). 
        \bibitem[5]{cite5} El fluido de transferencia de calor solar ingresa a la planta a $T_{f,in}=288^{\circ}C$. (s.f.). 
        \bibitem[6]{cite6} Se analizan y calculan las presiones en cada estado del ciclo, el trabajo específico de las turbinas y la bomba, la transferencia de calor en los diversos componentes (calentador, recalentador, condensador, intercambiador recuperativo), y la generación de entropía específica en cada componente principal. (s.f.). 
        \bibitem[7]{cite7} Se determina la eficiencia térmica global del ciclo y se verifica el cumplimiento de la primera y segunda ley de la termodinámica. (s.f.). 
        \bibitem[8]{cite8} Adicionalmente, se presenta un diagrama T-s del ciclo y se investiga la influencia de la presión de recalentamiento sobre la eficiencia de la planta, identificando el intervalo óptimo de esta presión. (s.f.). 
        \bibitem[9]{cite9} Finalmente, se analizan los resultados para identificar el equipo con mayor generación de entropía, lo que permite identificar áreas de mejora en el diseño y la operación del ciclo. (s.f.). 
        \bibitem[10]{cite10} ACTIVIDAD. (s.f.). 
        \bibitem[11]{cite11} Se considera la información propuesta que explica cada proceso termodinámico en los componentes del ciclo. (s.f.). 
        \bibitem[12]{cite12} Se examinan las tasas de transferencia de calor y de generación de entropía, tomando en cuenta las propiedades termodinámicas del tolueno. (s.f.). 
        \bibitem[13]{cite13} Los resultados evidencian beneficios en el uso de este fluido en comparación con el agua, particularmente en sistemas de recuperación de calor a temperaturas intermedias. (s.f.). 
        \bibitem[14]{cite14} Una planta de energía solar consiste en un ciclo Rankine que utiliza la energía solar como fuente de calor. (s.f.). 
        \bibitem[15]{cite15} Los receptores parabólicos concentran la energía solar en una tubería que transporta un fluido de transferencia de calor. (s.f.). 
        \bibitem[16]{cite16} Este fluido se calienta a medida que fluye a través del campo solar y luego ingresa a la planta de energía. (s.f.). 
        \bibitem[17]{cite17} El fluido transfiere calor al fluido de trabajo de la planta de energía para proporcionar la energía térmica que impulsa el ciclo de potencia. (s.f.). 
        \bibitem[18]{cite18} El fluido de transferencia de calor sale del campo y entra a la planta de energía a $T_{f,in}=288^{\circ}C$. (s.f.). 
        \bibitem[19]{cite19} Debido a que la temperatura de trabajo del ciclo es tan baja, el agua no es un fluido de trabajo muy eficiente. (s.f.). 
        \bibitem[20]{cite20} en cambio, se usa tolueno en el ciclo. El tolueno sale del calentador a: $T_{1}=T_{f,in}-\Delta T_{H}$ donde $\Delta T_{H}=20~K$ El tolueno se expande en la turbina de alta presión desde $P_{1}=P_{high}=1034~kPa$ hasta una presión intermedia $P_{2}=P_{reheat}=250~kPa$ con una eficiencia $\eta_{HPt}=0,81$. (s.f.). 
        \bibitem[21]{cite21} Luego, el fluido pasa a un recalentador donde se eleva a una nueva temperatura: $T_{3}=T_{f,in}-\Delta T_{RH}$ donde $\Delta T_{RH}=20~K$ El tolueno que sale de los recalentadores pasa por la turbina de baja presión que tiene una eficiencia de $\eta_{LPt}=0,78$. (s.f.). 
        \bibitem[22]{cite22} La presión de condensación se ajusta de modo que el tolueno que sale del condensador sea líquido saturado a: $T_{6}=$ $T_{amb}+\Delta T_{c}$ donde $T_{amb}=35^{\circ}C$ es la temperatura ambiente y $\Delta T_{c}=15~K.$ El líquido es bombeado de nuevo a alta presión por una bomba con eficiencia $\eta_{p}=0,6$. Parte del calor del tolueno que sale de la turbina de baja presión se recupera mediante un intercambiador de calor regenerativo, que precalienta el fluido antes de ingresar al calentador. (s.f.). 
        \bibitem[23]{cite23} El tolueno entra al condensador a una temperatura dada por: $T_{5}=T_{7}-\Delta T_{7}$ donde $\Delta T_{r}=20~K$. (s.f.). 
        \bibitem[24]{cite24} Se desprecia la caída de presión a través del calentador, el recalentador, el intercambiador de calor recuperativo y el condensador. (s.f.). 
        \bibitem[25]{cite25} Transferencia de calor en el calentador, recalentador y condensador. Para calcular las transferencias de calor específicas en el calentador principal $(q_{H})$, el recalentador $(q_{RH})y$ el condensador $(q_{C})$ en un ciclo Rankine regenerativo con recalentamiento, se emplean las diferencias de entalpía entre los estados en los que se lleva a cabo el intercambio térmico. (s.f.). 
        \bibitem[26]{cite26} $P_{1}=1034~kPa$ $P_{2}=250~kPa$ $P_{3}=250~kPa$ $q_{H}=h_{1}-h_{7}=727,394~kJ/kg$ $q_{RH}=h_{3}-h_{2}=67,463~kJ/kg$ $q_{C}=q_{8}-q_{5}=-639,679kJ/kg$. (s.f.). 
        \bibitem[27]{cite27} $P_{6}=12,13kPa$ $P_{7}=P_{6}$ es la salida de la bomba. $P_{8}=P_{1}=1034~kPa$ (entrada a la bomba). (s.f.). 
        \bibitem[28]{cite28} la eficiencia térmica de un ciclo de potencia indica la cantidad de energía térmica que realmente se transforma en trabajo útil. (s.f.). 
        \bibitem[29]{cite29} Teniendo en cuenta lo anterior, la eficiencia ( ) es: $\eta=\frac{W_{nsto}}{Q_{suministrado}}=$ $W_{HPt}+W_{LPt}-W_{F}$ Sustituyendo los valores: $\eta=$ $49,697年10883-2,030$ $727.394+67,463$ $\eta=0,1927~\eta=19,27\%$. (s.f.). 
        \bibitem[30]{cite30} El trabajo específico obtenido de la turbina de baja presión $W_{LPt}$, en kJ/kg y la generación de entropía específica en esta turbina $s_{gen,LPt}$, en kJ/kg K. El trabajo específico de la turbina a baja presión se calcula como la diferencia entre la entrada y la salida de entalpía: $W_{LPt}=h_{4}-h_{3}=105,483~kJ/kg$ La generación de entropía se calcula como la diferencia entre la salida y la entrada de entropía: $s_{gen}=s_{4}-s_{3}=0,06~kJ/kg\cdot K$. (s.f.). 
        \bibitem[31]{cite31} El trabajo específico requerido por la bomba $W_{P}$. en kJ/kg y la generación de entropía específica en la bomba $s_{gen,P}$ en $kJ/kg\cdot K.$ El trabajo requerido por la bomba nuevamente se calcula como la diferencia de la entrada y la salida de entalpía en la bomba: $W_{P}=h_{7}-h_{6}=2.030~kJ/kg$. Primera ley de la termodinámica: Es conocida como el principio de la conservación de la energía. (s.f.). 
        \bibitem[32]{cite32} La generación específica de entropía en la bomba es la entropía de entrada y salida de esta: $s_{gen,P}=s_{7}-s_{6}=0,006~kJ/kg\cdot K$. Establece que: (Energía total que entra al sistema) (Energía total que sale del sistema) = (Cambio de energía total del sistema) $\Delta E_{sistema}=$ $E_{entrada}-E_{salida}$ Como se está trabajando en un ciclo cerrado, podemos considerar que $\Delta h_{total}=0~y~Q_{neto}=W_{neto}$ Por tanto, tenemos: $\Sigma Q_{entra}-\sum Q_{sale}=\sum W_{sale}-\sum W_{entra}$. (s.f.). 
        \bibitem[33]{cite33} La transferencia de calor específica en el intercambiador de calor recuperativo $q_{r}$ en kJ/kg y la generación de entropía específica $s_{gen,r}$, en $kJ/kg\cdot K$. (s.f.). 
        \bibitem[34]{cite34} La transferencia de calor específica en el intercambiador de calor recuperativo se calcula como la diferencia de la entrada y salida de las entalpías de los estados en el que se encuentra: $q_{r}=h_{2}-h_{5}=641,709~kJ/kg$ La generación específica de entropía del recuperador se calcula a partir de la irreversibilidad del proceso: $s_{aen,r}=-1,837~kJ/kg\cdot K$. El trabajo específico de la turbina de alta presión se calcula como la diferencia de las entalpías que entran y salen: $W_{HPt}=h_{2}-h_{1}=49,697~kJ/kg)$. (s.f.). 
        \bibitem[35]{cite35} La generación de entropía se calcula como la diferencia de entropía de la generación que entra y sale: $s_{gen,HPt}=s_{2}-s_{1}=0,231~kJ/kg\cdot K$. Corroboramos que $Q_{neto}=W_{neto}:W_{neto}=W_{t}+$ $W_{t,lp}-W_{b}=49,69+105,48-0=155,17kJ/kg$. $Q_{neto}=Q_{entra}-(h_{4}-h_{5})-Q_{salida}=794,85-0-641,71=153,14~kJ/kg$ Se puede ver un pequeño margen de error, pero se puede corroborar que $W_{neto}\approx Q_{neto}$. (s.f.). 
        \bibitem[36]{cite36} Segunda ley de la termodinámica: Un proceso debe satisfacer tanto la primera como la segunda ley de la termodinámica para que pueda realizarse. (s.f.). 
        \bibitem[37]{cite37} Como es un ciclo cerrado ideal, la entropía debe regresar a su estado inicial, así que se debe cumplir que $\Delta S_{ciclo}=0$. (s.f.). 
        \bibitem[38]{cite38} Teniendo en cuenta las entropías halladas anteriormente, tenemos (observar cuadro III). (s.f.). 
        \bibitem[39]{cite39} Cuadro I: Datos termodinámicos de los estados del ciclo. (s.f.). 
        \bibitem[40]{cite40} The following table: "Estado ", "Temperatura (K) ", "Presión (kPa) Entalpía (kJ/kg) Entropía (kJ/kg-K) " "1 ", "541.15 ", "1034.00 614.74 1.298 " "2 ", "506.54 ", "1.321 250.00 565.05 " "3 ", "541.15 ", "250.00 632.51 1.450 " "4 ", "482.81 ", "12.29 527.03 1.513 " "5 ", "343.15 ", "12.29 527.03 1.513 " "6- 6 ", "323.15 ", "12.29 -114.68 -0.324 " "7 ", "323.15 ", "-0.318 12.29 -112.65 " "8 ", "323.15. ", "1034.00 -0.318 -112.65 ". (s.f.). 
        \bibitem[41]{cite41} Cuadro II: Análisis de la Primera Ley de la Termodinámica. (s.f.). 
        \bibitem[42]{cite42} The following table: "Proceso ", "Ecuación ", "Procedimiento (kJ/kg) " "W turbina HP ", "$W_{t}=h_{1}-h_{2}$ ", "614,74-565,05 $5,05=49,69$ " "Q recalentamiento ", "$Q_{r}=h_{3}-h_{2}$ ", "$632,51-565,05=67,46$ " "W turbina Lp ", "$W_{t,lp}=h_{3}-h_{4}$ ", "$632,51-527,03=105,48$ " "Q total añadido W bomba ideal ", "$Q_{entra}=(h_{1}-h_{8})+(h_{3}-h_{2}$) $W_{b}=h_{8}-h_{7}$ ", "$727,39+67,46=794,85$ $-112,65-(-112,65)=0$ " "W neto total ", "$W_{neto}=W_{t}+W_{t,l_{p}}-W_{b}$ ", "$49,69+105,48-0=155,17$ " "Q rechazado por condensador ", "$Q_{salida}=h_{5}-h_{6}$ ", "$527,03-(-114,68)=641,71$ ". (s.f.). 
        \bibitem[43]{cite43} Cuadro III: Análisis de la Segunda Ley de la Termodinámica $(\Delta s)$. Proceso. As (kJ/kg K). (s.f.). 
        \bibitem[44]{cite44} The following table: "1 y 2 ", "$\Delta s=1,321-1,298=0,023$ " "3 y4 ", "$\Delta s=1,513-1,450=0,063$ " "5 y 6 ", "$\Delta s=-0,324-1,513=-1,837$ " "6 y 7 7 y 8 ", "As-0,318(-0,324) 0,006 $\Delta s=-0,318-(-0,318)=0$ ". (s.f.). 
        \bibitem[45]{cite45} Figura 1: Diagrama T-s. Realizado en GeoGebra Clásico https://www.geogebra.org/classic?lang=es. (s.f.). 
        \bibitem[46]{cite46} CONCLUSIÓN. Los datos termodinámicos del ciclo Rankine con recalentamiento, presentados en la Tabla I, evidencian las condiciones de cada estado en términos de temperatura, presión, entalpía y entropía. (s.f.). 
        \bibitem[47]{cite47} Estos valores son fundamentales para el análisis energético y exergético del ciclo, permitiendo determinar los trabajos en turbinas y bombas, las transferencias de calor en los intercambiadores y, consecuentemente, la eficiencia térmica global del sistema. (s.f.). 
        \bibitem[48]{cite48} La coherencia de estos datos es crucial para verificar el cumplimiento de la primera y segunda ley de la termodinámica, como se detalla en las Tablas II y III, respectivamente, y para la construcción del diagrama T-s (Figura 1), que visualiza el comportamiento del ciclo. (s.f.). 
        \bibitem[49]{cite49} Çengel, Y. A., Boles, M. A., \& Kanoğlu, M. (2019). Termodinámica, 9ª ed. McGraw Hill. 
        \bibitem[50]{cite50} Moran, M. J., Shapiro, H. N., Boettner, D. D., \& Bailey, M. B. (2014). Fundamentals of Engineering Thermodynamics, 8ª ed. Wiley. 
    \end{thebibliography}
\end{frame}

\end{document}