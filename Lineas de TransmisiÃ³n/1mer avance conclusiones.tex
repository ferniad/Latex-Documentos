\section*{Conclusiones}
\begin{itemize}
    \item A partir del análisis comparativo realizado sobre los parámetros de diseño que establece la CREG 025 – 1995 y los descritos en la convocatoria de la UPME, se puede afirmar que cumplen con la normativa establecida en cuanto a la tensión, frecuencia, conductores, niveles de radiointerferencia, entre otros, garantizando de esta manera la confiabilidad de operación de nuestro sistema.\\
    \item Se concluye la importancia de tomar en cuenta los criterios socioeconómicos, los criterios físicos y los criterios de seguridad para la construcción y realización de la convocatoria, ya que por medio de estos se establecen las pautas requeridas, tales como minimizar los impactos en el medio ambiente y comunidades cercanas, así como las evaluaciones del terreno y el entorno para garantizar la confiabilidad del proyecto, entre otros.\\
    \item Con base en los requisitos del Reglamento Técnico de Instalaciones Eléctricas (RETIE) y a través de la observación directa de la salida de campo, donde se evidenciaron los principales componentes de la torre de transmisión, así como también el dimensionamiento y distancias de la misma, se concluye que esta cumple con los requisitos mínimos de distancias de seguridad, aisladores y puesta a tierra que se establecen.
\end{itemize}


