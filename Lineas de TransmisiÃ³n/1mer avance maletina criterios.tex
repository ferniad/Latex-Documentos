\section{Criterios Socioeconómicos}

\subsection{Población y Demografía}
Se consideran la distribución y densidad poblacional, así como la estructura de edad y género en las zonas de influencia. Estos datos permiten dimensionar la demanda de servicios, la disponibilidad de mano de obra y el posible impacto social.

\subsection{Tenencia de la Tierra y Restitución}
Se analiza la distribución de la propiedad, los tipos de tenencia (privada, colectiva, pública) y la existencia de procesos de restitución de tierras. Esto permite anticipar posibles conflictos de uso y definir estrategias para la gestión predial.

\subsection{Comunidades Étnicas y Actores Sociales}
Se identifican diferentes comunidades. El objetivo es garantizar la participación, el reconocimiento de sus derechos y la consideración de sus dinámicas culturales.

\subsection{Superposición con Otros Proyectos}
Se consideran iniciativas en curso o planificadas (infraestructura, hidrocarburos, minería, agricultura, etc.) para identificar sinergias y prevenir impactos acumulativos o conflictos en la ocupación del territorio.

\section{Criterios Físicos}

\subsection{Estabilidad del Terreno}
Se evalúan las características geológicas y estructurales para determinar posibles deformaciones en la corteza y la historia geológica del área. Asimismo, se consideran amenazas naturales como la sismicidad y la remoción en masa (deslizamientos, derrumbes, etc.), que pueden afectar la infraestructura y la seguridad en la zona de influencia.

\subsection{Contaminación y Erosión}
Se identifican los factores que podrían generar deterioro en la calidad del aire, del suelo y de las fuentes hídricas. De igual manera, se revisan los procesos de erosión y sedimentación que pueden incrementarse con las actividades de construcción, buscando medidas de control y prevención.

\subsection{Recursos Hídricos}
Incluye la revisión de la disponibilidad y calidad del agua, tanto superficial como subterránea. Se procura salvaguardar los cuerpos de agua y sus zonas de protección, asegurando la sostenibilidad del recurso y minimizando posibles impactos en los ecosistemas asociados.

\subsection{Uso del Suelo y Paisaje}
Se analizan las vocaciones y usos actuales del suelo, así como los lineamientos de ordenamiento territorial. El objetivo es integrar la infraestructura de manera armónica con el entorno, reduciendo impactos visuales y protegiendo la calidad del paisaje local.


%Criterios rabon Dai
\section*{Criterios de seguridad}
\begin{itemize}
    \item El trazado debe mantenerse alejado de áreas con amenaza de deslizamientos, inundaciones, fallas geológicas o incendios forestales, frecuentes en algunas zonas del Huila.
    
    \item La ruta debe permitir el acceso seguro para brigadas de mantenimiento y reparación sin exponer al personal a riesgos innecesarios.
    
    \item Las líneas de transmisión deben diseñarse y ubicarse considerando la protección con zonas habitadas, vías, cultivos, estructuras y el entorno, manteniendo distancias mínimas de seguridad que cumplan con las normativas eléctricas vigentes para minimizar riesgos de descargas, cortocircuitos o accidentes.
\end{itemize}