\newpage
\section{Parámetros de Diseño Según la CREG025 – 1995}
\subsection{Frecuencia}
El valor nominal de la frecuencia del SIN colombiano es de 60,00 Hz.

\subsection{Tensión}
La tensión nominal del STN es de 220 kV y 500 kV. No obstante, para efectos de diseño de nuevas instalaciones, se exige una tensión nominal de 230 kV.

\subsection{Longitud de la Línea de Transmisión}
En todas las actividades relacionadas con diseño, cálculo, tendido, estimación de materiales y construcción, se entiende que la línea de transmisión está comprendida entre los pórticos de salida de cada subestación que sirve de fijación al vano que las une a la primera torre.

\subsection{Conductores por Fase}
\begin{itemize}
    \item Resistencias eléctricas, medida en W/km a 20 grados C, debe ser igual o menor a la determinada por la UPME.
    \item Niveles de campos eléctricos y magnéticos:
    \begin{itemize}
        \item A borde de servidumbre = 5 kV/m
        \item Campo magnético = 1 Gauss
        \item Cruces de carreteras 10 a 12 kV/m
    \end{itemize}
    \item Niveles máximos de radio interferencia:
    \begin{itemize}
        \item Zonas rurales: 22 dB a 80 m del eje de la línea a 1000 kHz
        \item Zonas Urbanas: 22 dB a 40 m del eje de la línea a 1000 kHz
    \end{itemize}
    \item Cable de guarda: Todas las líneas de transmisión del STN deberán tener cable de guarda. Este deberá soportar el impacto directo de las descargas eléctricas atmosféricas.
    \item Aislamiento: Se define mediante combinación de las distancias mínimas, todo esto para evitar sobretensiones por descargas atmosféricas, de frecuencia industrial, etc.
\end{itemize}

\subsection{Comportamiento Mecánico del Conductor de Fase y Cable de Guarda}
En cualquier condición, la tensión longitudinal máxima en el conductor o cable de guarda no deberá exceder el 50\% de su correspondiente tensión de rotura.

\subsection{Estructuras}
El cálculo de estas, depende mucho del tipo de estructura, ya sea de suspensión, de retención o de terminal. También se asocian a si estos están en una condición normal o anormal, incluyendo parámetros de viento, temperatura, rotura de los conductores o cables de guarda, etc.

\subsection{Cimentaciones}
Las cimentaciones deberán resistir todas las hipótesis de carga que se estipulen para cada uno de los tipos de estructura.

\subsection{Localización de Estructuras}
La localización de las estructuras deberá cumplir con ciertos requisitos para la seguridad sobre el terreno y obstáculos, entre estos se encuentran las carreteras principales, árboles, cercas, ferrocarriles, ríos navegables, embalses, entre otros.

\subsection{Cadenas de Aisladores y Herrajes}
Los aisladores deberán ser fabricados en porcelana, vidrio o poliméricos.

\subsection{Puesta a Tierra}
Este sistema de puesta a tierra debe estar en cada estructura y ser diseñado según las condiciones específicas de las líneas.

\subsection{Servidumbres}
El ancho de la faja de servidumbre requerida será establecido por el propietario de la línea, ajustado con base en los niveles de campo electromagnético y de radio interferencia.
