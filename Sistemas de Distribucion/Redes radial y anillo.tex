
\documentclass[conference]{IEEEtran}

% --- Idioma y Codificación (Se cargan primero) ---
\usepackage[utf8]{inputenc}
\usepackage[spanish]{babel}

% --- Matemáticas, Formato y Estructuras Avanzadas ---
% Paquetes matemáticos y de formato avanzado agrupados:
\usepackage{amssymb, amsmath, amsthm, empheq, bm}
% Paquetes para estructuras enmarcadas y colores específicos:
\usepackage{mdframed, float}
\usepackage{color, colortbl, xcolor}
% --- Figuras, Tablas y Gráficos ---
% Paquetes para imágenes, tablas y gráficos agrupados:
\usepackage{graphicx, subcaption, caption, booktabs, multirow, psfrag}
\usepackage{pgfplots} % Se mantiene aparte para mejor visibilidad
% --- Hipervínculos y Documentación ---
\usepackage{hyperref, bookmark}
% --- AJUSTE CRÍTICO: La clase IEEEtran ya maneja los márgenes.
% **Se recomienda ENCARECIDAMENTE comentar o eliminar esta línea**
% para mantener el formato oficial de la conferencia.
% \usepackage{geometry}
% \geometry{a4paper, margin=1in}


%%%%%%%%%%%%%%%%%%%%%%%%%%%%%%%%%%%%%%%%%%%%%%%%%%%%%%%%%%%%%%%%%%%%%%%%%%%%%%%

% Other Settings

%%%%%%%%%%%%%%%%%%%%%%%%%% Define some useful colors %%%%%%%%%%%%%%%%%%%%%%%%%%
% Definición de colores (mantener separados para claridad en la configuración)
\definecolor{lightgreen}{HTML}{A9D18E}
\definecolor{lightred}{HTML}{F4C7C3}
\definecolor{ocre}{RGB}{243,102,25}
\definecolor{mygray}{RGB}{243,243,244}
\definecolor{deepGreen}{RGB}{26,111,0}
\definecolor{shallowGreen}{RGB}{235,255,255}
\definecolor{deepBlue}{RGB}{61,124,222}
\definecolor{shallowBlue}{RGB}{235,249,255}
%%%%%%%%%%%%%%%%%%%%%%%%%%%%%%%%%%%%%%%%%%%%%%%%%%%%%%%%%%%%%%%%%%%%%%%%%%%%%%%

%%%%%%%%%%%%%%%%%%%%%%%%%% Define an orangebox command %%%%%%%%%%%%%%%%%%%%%%%%
\newcommand\orangebox[1]{\fcolorbox{ocre}{mygray}{\hspace{1em}#1\hspace{1em}}}
%%%%%%%%%%%%%%%%%%%%%%%%%%%%%%%%%%%%%%%%%%%%%%%%%%%%%%%%%%%%%%%%%%%%%%%%%%%%%%%

%%%%%%%%%%%%%%%%%%%%%%%%%%%% Spanish Environments %%%%%%%%%%%%%%%%%%%%%%%%%%%%%
% Definición de estilos y entornos
\newtheoremstyle{mytheoremstyle}{3pt}{3pt}{\normalfont}{0cm}{\rmfamily\bfseries}{}{1em}{{\color{black}\thmname{#1}~\thmnumber{#2}}\thmnote{\,--\,#3}}
\newtheoremstyle{myproblemstyle}{3pt}{3pt}{\normalfont}{0cm}{\rmfamily\bfseries}{}{1em}{{\color{black}\thmname{#1}~\thmnumber{#2}}\thmnote{\,--\,#3}}
\theoremstyle{mytheoremstyle}
\newmdtheoremenv[linewidth=1pt,backgroundcolor=shallowGreen,linecolor=deepGreen,leftmargin=0pt,innerleftmargin=20pt,innerrightmargin=20pt,]{theorem}{Theorem}[section]
\theoremstyle{mytheoremstyle}
\newmdtheoremenv[linewidth=1pt,backgroundcolor=shallowBlue,linecolor=deepBlue,leftmargin=0pt,innerleftmargin=20pt,innerrightmargin=20pt,]{definition}{Definition}[section]
\theoremstyle{myproblemstyle}
\newmdtheoremenv[linecolor=black,leftmargin=0pt,innerleftmargin=10pt,innerrightmargin=10pt,]{problem}{Problem}[section]
%%%%%%%%%%%%%%%%%%%%%%%%%%%%%%%%%%%%%%%%%%%%%%%%%%%%%%%%%%%%%%%%%%%%%%%%%%%%%%%

%%%%%%%%%%%%%%%%%%%%%%%%%%%%%%% Plotting Settings %%%%%%%%%%%%%%%%%%%%%%%%%%%%%
\usepgfplotslibrary{colorbrewer}
\pgfplotsset{width=8cm,compat=1.9}
%%%%%%%%%%%%%%%%%%%%%%%%%%%%%%%%%%%%%%%%%%%%%%%%%%%%%%%%%%%%%%%%%%%%%%%%%%%%%%%

%%%%%%%%%%%%%%%%%%%%%%%%%%%%%%% Title & Author %%%%%%%%%%%%%%%%%%%%%%%%%%%%%%%%


\author{\IEEEauthorblockN{Daniel Fernando Aranda Contreras}
\IEEEauthorblockA{Escuela E3T, Universidad Industrial de Santander\\
Correo electrónico: daniel2221648@correo.uis.edu.co}}
%%%%%%%%%%%%%%%%%%%%%%%%%%%%%%%%%%%%%%%%%%%%%%%%%%%%%%%%%%%%%%%%%%%%%%%%%%%%%%%

\begin{document}
    % Título
    \title{\uppercase{Análisis Comparativo de Redes de Distribución Eléctrica: Radial vs. En Anillo}}
    \maketitle
    \date{\today}
    \begin{abstract}
    El presente documento tiene como finalidad establecer una comparación detallada entre las dos topologías fundamentales de redes de distribución eléctrica: la configuración radial y la configuración en anillo (o bucle). Se analizan sus definiciones estructurales, se contrastan sus ventajas, desventajas, y se detallan sus usos más comunes, prestando especial atención al impacto de cada diseño en la fiabilidad del servicio y el costo operativo.    \end{abstract}

    \begin{IEEEkeywords}
    Distribución Eléctrica; Red Radial; Red en Anillo (Loop); Topología; Continuidad de Servicio; Fiabilidad; Costo/Economía; Manejo de Fallas; Cargas Críticas; Simplicidad Operativa.
    \end{IEEEkeywords}

    \section{Introducción}

La distribución de energía eléctrica es un componente vital de cualquier sistema de potencia, siendo el enlace final entre la fuente de generación y el consumidor. La topología de la red de distribución es un factor determinante en la calidad, la continuidad del servicio y el costo de la infraestructura. Históricamente, han prevalecido dos configuraciones primarias para los alimentadores: la radial y la en anillo (o bucle).

La elección entre una y otra no es trivial y depende de las prioridades de diseño, como la densidad de carga, la importancia crítica de los usuarios finales (e.g., hospitales, industrias) y las restricciones presupuestarias. A continuación, se definen y comparan estos dos enfoques de diseño para comprender cómo influyen en el funcionamiento global de la red.

\section{Objetivos}

El objetivo principal de este documento es \textbf{definir, comparar y analizar las características operativas y de diseño}, así como las ventajas y desventajas inherentes, de los sistemas de distribución eléctrica con configuración radial y en anillo.

\section{Desarrollo y Análisis Comparativo de Redes}

\subsection{Definición de Redes de Distribución Radial y en Anillo}

\subsubsection{Red de Distribución Radial}
Una red de distribución radial (Radial Feeder System) se caracteriza por presentar un \textbf{solo camino simultáneo} para el flujo de potencia hacia la carga. Las líneas se extienden desde la subestación de manera similar a las ramas de un árbol (configuración dendrítica). Si el camino se interrumpe, el resultado es la pérdida total del servicio para el cliente ubicado ``río abajo''.

\subsubsection{Red de Distribución en Anillo (Loop)}
Una red de distribución en anillo (Loop Feeder System) está diseñada para ofrecer \textbf{más de un camino simultáneo} para el flujo de potencia entre las fuentes de energía y cada cliente. Aunque están construidos como un bucle, a menudo se operan como un bucle abierto (open loop) con un interruptor normalmente abierto cerca del medio, manteniendo operativamente un circuito radial, pero con la capacidad de reconfiguración.

\subsection{Ventajas, Desventajas y Usos Comunes}

La siguiente tabla resume las principales características, contrastando ambos diseños:



\begin{table*}[t]
  \centering
  \caption{Comparación de Características: Red Radial vs. Red en Anillo}
  \label{tab:comparacion_redes_mejorada}
  \begin{tabular}{p{0.3\textwidth} p{0.3\textwidth} p{0.3\textwidth}}
    \toprule
    \textbf{Característica} & \textbf{Red Radial (Radial)} & \textbf{Red en Anillo (Loop)} \\
    \midrule
    \textbf{Ventajas} &
    \begin{itemize}
      \item Costo más bajo y simple de instalar.
      \item Sencillo de operar y económico.
      \item Predicción y control más fáciles de los flujos de potencia y voltaje.
      \item Protección de corriente de falla más sencilla.
    \end{itemize} &
    \begin{itemize}
      \item Mayor fiabilidad y continuidad del servicio.
      \item Cualquier punto puede ser alimentado desde dos direcciones.
      \item Las fallas interrumpen el servicio a un segmento pequeño o a ningún cliente (redirección).
      \item El impacto de una falla se reduce al mínimo (cambio de voltaje ligero) cuando se opera cerrado.
    \end{itemize} \\
    \cmidrule(lr){1-3}
    \textbf{Desventajas} &
    \begin{itemize}
      \item Menos fiable que los sistemas en anillo o malla.
      \item Una falla en cualquier punto deja fuera de servicio a todos los clientes río abajo.
      \item Los alimentadores primarios radiales suelen ser los responsables de la falta de continuidad.
    \end{itemize} &
    \begin{itemize}
      \item Más costoso que el arreglo radial.
      \item Requiere más equipo y el conductor debe ser más grande para manejar la carga desde cualquiera de los dos extremos (doble capacidad).
      \item Ligeramente más complicado de analizar y operar.
    \end{itemize} \\
    \cmidrule(lr){1-3}
    \textbf{Usos Comunes} &
    \begin{itemize}
      \item Diseño más utilizado (más del 99\% en Norteamérica).
      \item Zonas Habitacionales y Comerciales consideradas no muy importantes.
      \item Cargas Ligeras y áreas de carga de densidad media.
      \item Uso en la mayoría de los circuitos de distribución primaria y secundaria.
    \end{itemize} &
    \begin{itemize}
      \item Cargas Críticas (e.g., hospitales) donde la continuidad es crucial.
      \item Grandes Cargas Urbanas o edificios medianos/grandes.
      \item Distribución Subterránea Urbana (UG practice) como regla.
      \item Utilizado en URD para una restauración de servicio más rápida tras una falla.
    \end{itemize} \\
    \bottomrule
  \end{tabular}
\end{table*}


\subsection{Diferencias Fundamentales}

Las diferencias operacionales y estructurales se resumen en cuatro puntos clave:

\begin{enumerate}
\item \textbf{Topología y Caminos de Flujo:} La red radial posee una \textbf{única trayectoria}, mientras que la red en anillo tiene \textbf{al menos dos trayectorias} de suministro entre la fuente y la carga.
\item \textbf{Costo y Capacidad:} El diseño \textbf{Radial} es el de menor costo y el más simple. El diseño \textbf{Anillo} es más costoso, ya que requiere conductores y equipos de mayor capacidad para soportar la carga desde cualquiera de los dos extremos.
\item \textbf{Manejo de Fallas y Fiabilidad:} En un sistema \textbf{Radial}, una falla interrumpe a todos los clientes aguas abajo, requiriendo reconfiguración manual. En el sistema \textbf{Anillo}, la fiabilidad es mucho mayor: la sección con falla se aísla rápidamente, y el servicio se mantiene alimentando al resto del circuito desde la dirección alternativa.
\item \textbf{Complejidad de Análisis:} Los flujos de potencia \textbf{Radiales} son predecibles, simplificando el análisis. La red \textbf{Anillo}, si se opera como bucle cerrado, requiere técnicas de red más complejas para el análisis de flujos, fallas y protección.
\end{enumerate}

\section{Conclusiones}

La decisión de implementar una red de distribución eléctrica radial o en anillo es un balance directo entre el costo inicial y la continuidad del servicio deseada. La red \textbf{Radial} se ha consolidado como la opción predominante a nivel global debido a su \textbf{simplicidad operativa y bajo costo}. Sin embargo, esta topología sacrifica la fiabilidad, ya que cualquier falla resulta en una interrupción inevitable del servicio para los clientes aguas abajo.

Por otro lado, la red \textbf{En Anillo} emerge como la solución ideal para \textbf{cargas críticas y áreas de alta densidad} donde la continuidad del servicio es primordial. Aunque requiere una inversión inicial significativamente mayor y un análisis más complejo, su capacidad de aislar fallas y reconfigurar el servicio de manera automática o semiautomática garantiza una alta resiliencia. En la práctica moderna, es común que los sistemas radiales incluyan lazos de interconexión con otros circuitos (normally open ties) para aumentar su flexibilidad, combinando la economía del diseño radial con un nivel básico de redundancia.

    
    \nocite{*} % Asegura que todas las entradas de la bibliografía se incluyan, incluso si no se citan directamente
    \bibliographystyle{IEEEtran}
    \bibliography{ref1} % Asegúrate de que tu archivo de bibliografía se llama 'ref.bib'

\end{document}