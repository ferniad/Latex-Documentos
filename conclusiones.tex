\section{Comparación de procesos de muestreo}
    Para los casos en los cuales se emplearon frecuencias de muestreo de 240 Hz y 180 Hz son los mas eficaces puesto que solo emplean una sola ventana de observación para realizar la estimación de los parámetros de la señal, en cambio en las señales de 280 Hz y 200 Hz se realizan de otra ventana de observación incluyendo las dos en las que se tomo la medición además requerían de 5 y 4 medidas adicionales respectivamente.\\ 

\section{Relación con el Teorema de Nyquist-Shannon}
El teorema de Nyquist-Shannon establece que, para evitar el aliasing y permitir la reconstrucción perfecta de una señal, la frecuencia de muestreo debe ser al menos el doble de la frecuencia máxima de la señal.
\begin{equation}
    f_s \geq 2 f_{\text{máx}}
\end{equation}

\section{Conclusiones y Recomendaciones}
La frecuencia de muestreo tiene un impacto significativo en la estimación de parámetros en sistemas eléctricos. La correcta elección de esta frecuencia, fundamentada en el teorema de Nyquist-Shannon, asegura que la información crítica de la señal sea capturada de manera precisa y que se minimicen los efectos del aliasing. En conclusión, para obtener mediciones confiables y precisas, es esencial seleccionar una frecuencia de muestreo que no solo cumpla con la condición de Nyquist, sino que también considere la naturaleza y dinámica del sistema en estudio.

