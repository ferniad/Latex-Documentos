
%%%%%%%%%%%%%%%%%%%%%%%%%%%%% Define Article %%%%%%%%%%%%%%%%%%%%%%%%%%%%%%%%%%
\documentclass[conference]{IEEEtran}
%%%%%%%%%%%%%%%%%%%%%%%%%%%%%%%%%%%%%%%%%%%%%%%%%%%%%%%%%%%%%%%%%%%%%%%%%%%%%%%

%%%%%%%%%%%%%%%%%%%%%%%%%%%%% Using Packages %%%%%%%%%%%%%%%%%%%%%%%%%%%%%%%%%%
\usepackage{geometry}
\usepackage{graphicx}
\usepackage{amssymb}
\usepackage{amsmath}
\usepackage{amsthm}
\usepackage{empheq}
\usepackage{mdframed}
\usepackage{booktabs}
\usepackage{lipsum}
\usepackage{graphicx}
\usepackage{color}
\usepackage{psfrag}
\usepackage{pgfplots}
\usepackage{bm}
\usepackage[spanish]{babel}
\usepackage[utf8]{inputenc} % Codificación UTF-8
\usepackage{amsmath}        % Soporte para ecuaciones matemáticas
\usepackage{graphicx}       % Manejo de imágenes
\usepackage{hyperref}       % Hipervínculos
\usepackage{caption}        % Formato para figuras
\usepackage{multirow}
\usepackage{subcaption}
\usepackage{biblatex}
\usepackage{csquotes}
\usepackage{bookmark}
%%%%%%%%%%%%%%%%%%%%%%%%%%%%%%%%%%%%%%%%%%%%%%%%%%%%%%%%%%%%%%%%%%%%%%%%%%%%%%%

% Other Settings

%%%%%%%%%%%%%%%%%%%%%%%%%% Page Setting %%%%%%%%%%%%%%%%%%%%%%%%%%%%%%%%%%%%%%%
\geometry{a4paper, margin=1in}

%%%%%%%%%%%%%%%%%%%%%%%%%% Define some useful colors %%%%%%%%%%%%%%%%%%%%%%%%%%
\definecolor{ocre}{RGB}{243,102,25}
\definecolor{mygray}{RGB}{243,243,244}
\definecolor{deepGreen}{RGB}{26,111,0}
\definecolor{shallowGreen}{RGB}{235,255,255}
\definecolor{deepBlue}{RGB}{61,124,222}
\definecolor{shallowBlue}{RGB}{235,249,255}
%%%%%%%%%%%%%%%%%%%%%%%%%%%%%%%%%%%%%%%%%%%%%%%%%%%%%%%%%%%%%%%%%%%%%%%%%%%%%%%

%%%%%%%%%%%%%%%%%%%%%%%%%% Define an orangebox command %%%%%%%%%%%%%%%%%%%%%%%%
\newcommand\orangebox[1]{\fcolorbox{ocre}{mygray}{\hspace{1em}#1\hspace{1em}}}
%%%%%%%%%%%%%%%%%%%%%%%%%%%%%%%%%%%%%%%%%%%%%%%%%%%%%%%%%%%%%%%%%%%%%%%%%%%%%%%

%%%%%%%%%%%%%%%%%%%%%%%%%%%% English Environments %%%%%%%%%%%%%%%%%%%%%%%%%%%%%
\newtheoremstyle{mytheoremstyle}{3pt}{3pt}{\normalfont}{0cm}{\rmfamily\bfseries}{}{1em}{{\color{black}\thmname{#1}~\thmnumber{#2}}\thmnote{\,--\,#3}}
\newtheoremstyle{myproblemstyle}{3pt}{3pt}{\normalfont}{0cm}{\rmfamily\bfseries}{}{1em}{{\color{black}\thmname{#1}~\thmnumber{#2}}\thmnote{\,--\,#3}}
\theoremstyle{mytheoremstyle}
\newmdtheoremenv[linewidth=1pt,backgroundcolor=shallowGreen,linecolor=deepGreen,leftmargin=0pt,innerleftmargin=20pt,innerrightmargin=20pt,]{theorem}{Theorem}[section]
\theoremstyle{mytheoremstyle}
\newmdtheoremenv[linewidth=1pt,backgroundcolor=shallowBlue,linecolor=deepBlue,leftmargin=0pt,innerleftmargin=20pt,innerrightmargin=20pt,]{definition}{Definition}[section]
\theoremstyle{myproblemstyle}
\newmdtheoremenv[linecolor=black,leftmargin=0pt,innerleftmargin=10pt,innerrightmargin=10pt,]{problem}{Problem}[section]
%%%%%%%%%%%%%%%%%%%%%%%%%%%%%%%%%%%%%%%%%%%%%%%%%%%%%%%%%%%%%%%%%%%%%%%%%%%%%%%

%%%%%%%%%%%%%%%%%%%%%%%%%%%%%%% Plotting Settings %%%%%%%%%%%%%%%%%%%%%%%%%%%%%
\usepgfplotslibrary{colorbrewer}
\pgfplotsset{width=8cm,compat=1.9}
%%%%%%%%%%%%%%%%%%%%%%%%%%%%%%%%%%%%%%%%%%%%%%%%%%%%%%%%%%%%%%%%%%%%%%%%%%%%%%%

%%%%%%%%%%%%%%%%%%%%%%%%%%%%%%% Title & Author %%%%%%%%%%%%%%%%%%%%%%%%%%%%%%%%
\author{\IEEEauthorblockN{Brayan Joanne Ballesteros Meza, Brayhan Steven Delgado Rueda, Daniel Fernando Aranda Contreras,\\ Jonathan Stiven Murcia Suarez}
\IEEEauthorblockA{Escuela E3T, Universidad Industrial de Santander\\
Correo electrónico: \{brayan2222069, brayan2212088, daniel2221648, jonathan2225092\}@correo.uis.edu.co}}

%%%%%%%%%%%%%%%%%%%%%%%%%%%%%%%%%%%%%%%%%%%%%%%%%%%%%%%%%%%%%%%%%%%%%%%%%%%%%%%
    \begin{document}
        % Título
        \title{\uppercase{Muestreo de señales eléctricas de tensión y corriente}}
        \maketitle
        % Resumen
        % Palabras clave        
        \begin{IEEEkeywords}
            Muestreo, 
            Señales eléctricas, 
            Tensión, 
            Corriente, 
            Frecuencia fundamental, 
            Mediciones, 
            Potencia activa, 
            Potencia reactiva, 
            Factor de potencia, 
            Luminarias, 
            Análisis de datos, 
            Escenarios, 
            Comparación de parámetros, 
            Muestras, 
            Eficaz, 
            Máximo, 
            Procesamiento de señales  
        \end{IEEEkeywords}

        \title{\uppercase{Muestreo de señales eléctricas de tensión y corriente en luminarias LED, incandescentes}}
        \section*{Parámetros de Medición Luminaria LED}
        \begin{align*}
        \text{Vimax} & = 169.8960 \\
        \text{V2max} & = 169.5800 \\
        \text{Vmax} & = 162.4910 \\
        \text{i1max} & = 0.6710 \\
        \text{i2max} & = 0.5840 \\
        \text{i3max} & = 0.8830 \\
        N & = 128 \\
        \end{align*}

        \begin{table}[h]
        \centering
        \caption{Mediciones de la primera serie para 128 muestras $(LED)$}
        \begin{tabular}{@{}ccccccc@{}}
        \toprule
        \textbf{Muestra} & \textbf{Vrms} & \textbf{Irms} & \textbf{P} & \textbf{Q} & \textbf{S} & \textbf{fp} \\ \midrule
        1 & 120.0987 & 0.2063 & 13.9468 & 20.4790 & 24.7770 & 0.5629 \\
        2 & 120.1127 & 0.2130 & 14.3059 & 21.2116 & 25.5850 & 0.5592 \\
        3 & 120.0256 & 0.2371 & 14.2739 & 24.6262 & 28.4639 & 0.5015 \\ \bottomrule
        \end{tabular}
        \end{table}

        \vspace{1cm}

        \begin{table}[h]
        \centering
        \caption{Mediciones de la segunda serie para 192 muestras $(LED)$}
        \begin{tabular}{@{}ccccccc@{}}
        \toprule
        \textbf{Muestra} & \textbf{Vrms} & \textbf{Irms} & \textbf{P} & \textbf{Q} & \textbf{S} & \textbf{fp} \\ \midrule
        1 & 120.0988 & 0.2063 & 13.9566 & 20.4744 & 24.7788 & 0.5632 \\
        2 & 120.1147 & 0.2131 & 14.3265 & 21.2074 & 25.5930 & 0.5598 \\
        3 & 120.0264 & 0.2372 & 14.2908 & 24.6169 & 28.4643 & 0.5021 \\ \bottomrule
        \end{tabular}
        \end{table}

        %%%%INCADESCENTES%%%%


        \section*{Parámetros de Luminarias Incandescentes}
        \begin{align*}
        \text{Vmax} & = 169.4750 \\
        \text{V2max} & = 169.0540 \\
        \text{V3max} & = 162.1750 \\
        \text{i1max} & = 0.8550 \\
        \text{i2max} & = 0.8510 \\
        \text{i3max} & = 0.8230 \\
        N & = 128 \\
        \end{align*}

        \begin{table}[h]
        \centering
        \caption{Mediciones de la primera serie para 128 muestras $(Incandescentes)$}
        \begin{tabular}{@{}ccccccc@{}}
        \toprule
        \textbf{Muestra} & \textbf{Vrms} & \textbf{Irms} & \textbf{P} & \textbf{Q} & \textbf{S} & \textbf{fp} \\ \midrule
        1 & 119.8105 & 0.6036 & 72.3210 & 0.5540 & 72.3231 & 1.0000 \\
        2 & 119.8353 & 0.6040 & 72.3754 & 0.6479 & 72.3783 & 1.0000 \\
        3 & 119.8378 & 0.6040 & 72.3728 & 0.7762 & 72.3770 & 0.9999 \\ \bottomrule
        \end{tabular}
        \end{table}

        \newpage

        \subsection*{Mediciones de la segunda serie para 192 muestras $(Incandescentes)$}
        \begin{table}[h]
        \centering
        \caption{Datos de la segunda serie}
        \begin{tabular}{@{}ccccccc@{}}
        \toprule
        \textbf{Muestra} & \textbf{Vrms} & \textbf{Irms} & \textbf{P} & \textbf{Q} & \textbf{S} & \textbf{fp} \\ \midrule
        1 & 120.0987 & 0.2063 & 13.9468 & 20.4790 & 24.7770 & 0.5629 \\
        2 & 120.1127 & 0.2130 & 14.3059 & 21.2116 & 25.5850 & 0.5592 \\
        3 & 120.0264 & 0.2372 & 14.2908 & 24.6169 & 28.4643 & 0.5021 \\ \bottomrule
        \end{tabular}
        \end{table}


        \section{Análisis y Comparación de Parámetros en Luces LED e Incandescentes}  
        Al comparar los valores de tensión y corriente RMS obtenidos en los incisos previos, se observa que ambas luces presentan niveles de tensión RMS similares. Pero por otra parte, las lámparas incandescentes registran valores de corriente RMS mayores en comparación con las luces LED. Además, si analizamos las potencias de dimensionamiento, las luces incandescentes presentan mayor magnitud en comparación con las luces LED. Las tablas provenientes de los puntos 1-3 muestran que las lámparas incandescentes operan con un factor de potencia unitario, debido a su comportamiento resistivo. En contraste, las luces LED presentan un factor de potencia considerablemente bajo, lo que implica una mayor presencia de potencia reactiva en su consumo. En conclusión, si buscamos un buen aprovechamiento de la energía, son una mejor opción las luces incandescentes. Pero por otra parte, si queremos una magnitud menor de potencia de dimensionamiento y menor consumo de corriente, son una mejor opción las luces LED.
        \\
        \section{Análisis Comparativo de Parámetros de Señales en Luminarias: Similitudes y Diferencias en Escenarios de Tensión y Corriente}
        Al analizar los datos obtenidos de 128 muestras en los ítems 2 y 3 en comparación con las 192 muestras del ítem 5, se observan ligeras variaciones atribuibles al número de muestras. Las 128 muestras representan un ciclo completo de la señal a 60 Hz, con una frecuencia de muestreo de 7860 Hz, mientras que las 192 muestras abarcan 1.5 ciclos, lo que implica un análisis incompleto de los ciclos. Los valores eficaces de tensión y corriente son similares en ambas mediciones, con ligeras diferencias por la cantidad de muestras. Los valores máximos se mantienen constantes, ya que dependen de la amplitud máxima de la señal. En contraste, la potencia activa, reactiva y aparente presenta variaciones menores debido a la falta de un ciclo completo, mientras que el factor de potencia se mantiene prácticamente constante. Aunque se muestreó originalmente a 7680 Hz, el vector se recortó a 128 muestras para obtener un ciclo. Con 192 muestras (un ciclo y medio), se observan diferencias mínimas en los datos de las nuevas tablas para luminarias LED e incandescentes.
        
        \end{document}  
