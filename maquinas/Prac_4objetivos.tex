\section{Objetivos} 
\begin{itemize}
    \item Realizar las pruebas de vacío y de cortocircuito en un transformador monofásico configurado como autotransformador para analizar su comportamiento eléctrico.
    \item Determinar experimentalmente el rendimiento y la regulación del autotransformador bajo diferentes tipos de carga: resistiva, inductiva y capacitiva.
\end{itemize}

\section{Equiops y materiales}
\begin{itemize}
    \item Transformador monofásico.
    \item Voltímetro CA.
    \item Amperímetro CA.
    \item Vatímetro monofásico CA.
    \item Transformador de corriente (CT).
    \item Cargas: resistivas (R), inductivas (L) y capacitivas (C).
\end{itemize}

\section{Introducción}
\text{Los autotransformadores son dispositivos eléctricos que facilitan la modificación de los niveles de tensión dentro de un rango específico de manera eficiente. A diferencia de los transformadores tradicionales, en los autotransformadores los devanados primario y secundario no están completamente aislados, ya que comparten una parte del mismo arrollamiento. Esto resulta en una reducción del material conductor utilizado y en una mejora de la eficiencia del sistema. En este laboratorio, se realizará un estudio de un transformador monofásico funcionando como autotransformador, a través de pruebas experimentales que permitirán evaluar su rendimiento y regulación en diversas condiciones de carga.}

\section{Conclusión}
\text{A lo largo del laboratorio, se estudió el comportamiento de un transformador monofásico operando como autotransformador, evaluando su rendimiento y regulación bajo diferentes condiciones de carga, las cuales tienen una gran influencia en su rendimiento depediendo el tipo de carga, si es una carga RC conectada en paralelo tiende a aunmentar su tension de salida en comparacion con una carga netamente resistiva, mientras que con una carga inductiva conectada en serie su nivel de tension de salida disminuye, pero si conectamos una carga RLC está tiene un comportamiento de compensacion, si las cargas inductivas (L) y capacitivas (C) tienen valores similares.}