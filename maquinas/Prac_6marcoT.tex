\section{MARCO TEÓRICO}
Las máquinas de corriente continua (CC) son dispositivos electromecánicos que convierten energía eléctrica en energía mecánica (motores) o viceversa (generadores). Aunque en la actualidad han sido en gran parte reemplazadas por máquinas de corriente alterna en muchas aplicaciones, las máquinas de CC siguen siendo fundamentales en entornos industriales, sistemas de control, vehículos eléctricos y laboratorios educativos, debido a su capacidad de control preciso de velocidad y torque.\newline

Una máquina de corriente continua está compuesta principalmente por un \textbf{estator}, que proporciona el campo magnético, y un \textbf{rotor} o \textbf{armadura}, que gira dentro de ese campo. El intercambio de corriente entre la armadura giratoria y el circuito externo se logra mediante el \textbf{conmutador} y las \textbf{escobillas}, lo que permite mantener una dirección unidireccional de la corriente en el circuito externo, incluso cuando la máquina está en movimiento.\newline

Las \textbf{medidas preliminares} en el estudio de una máquina de CC son esenciales para garantizar un funcionamiento seguro y eficaz antes de realizar ensayos más exigentes. Estas medidas incluyen la \textbf{resistencia de los devanados del inducido y del campo}, la \textbf{verificación de la polaridad}, la \textbf{identificación de terminales}, la \textbf{comprobación de la continuidad}, y la \textbf{inspección visual de conexiones y escobillas}. Estas pruebas permiten detectar fallas como cortocircuitos, circuitos abiertos o conexiones incorrectas, que podrían afectar el rendimiento de la máquina o provocar daños durante su operación.\newline

Comprender el comportamiento eléctrico básico de la máquina en estado de reposo (sin carga) permite también establecer una línea base para futuros análisis de desempeño, eficiencia y respuesta dinámica. Además, estas mediciones son cruciales para el desarrollo de modelos matemáticos que representen la máquina en simulaciones o aplicaciones de control.\newline



