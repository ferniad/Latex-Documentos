\section{Conclusiones}
\label{sec:conclusiones}

\begin{itemize}
    \item La identificación correcta de los bornes en máquinas tipo Shunt y Compound permitió una conexión segura y adecuada para las pruebas posteriores, evitando errores comunes que podrían provocar fallas o daños en el equipo.
    \item La medición directa de las resistencias de los devanados permitió comprobar la continuidad de los bobinados y diferenciar entre el devanado de campo shunt, de mayor resistencia, y el de armadura, de menor resistencia, lo cual facilitó su identificación.
    \item La prueba de resistencia de aislamiento arrojó valores aceptables (mayores a 1 [M$\Omega$]), indicando que los devanados no presentan fugas peligrosas de corriente hacia tierra, y que los materiales aislantes se encuentran en buen estado.
    \item La experiencia práctica facilitó la comprensión de la estructura interna y el comportamiento eléctrico básico de las máquinas de corriente continua, fortaleciendo el vínculo entre teoría y práctica en la ingeniería eléctrica.
    \item La caída de tensión medida en las escobillas fue de 0,415 V, lo que está dentro de los límites esperados, indicando un buen contacto y mínima pérdida de energía.
    \item Se destacó la importancia de la zona neutra geométrica, que es el punto donde las escobillas deben estar ubicadas para evitar chispas y desgaste en el conmutador. Su correcta colocación es vital para que la máquina funcione de manera eficiente y segura.
    \item Las resistencias en las máquinas Shunt y Compound fueron muy diferentes, lo cual es esperado debido a sus distintas configuraciones y funciones.
    \item Las pruebas iniciales como la identificación de bornes y medición de resistencias son claves para asegurar el correcto funcionamiento y evitar fallos de la máquina.
\end{itemize}
