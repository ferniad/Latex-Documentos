\section{Introducción}
Los transformadores se han convertido en parte esencial de nuestro día a día, los podemos encontrar en muchos lados, incluso en nuestras casas, estos dispositivos permiten hacer el cambio de tensión para facilitar el manejo de esta, podemos encontrarlo en cargadores cuya función es reducir la tensión para el correcto funcionamiento y evitando el daño del dispositivo a cargar, se encuentran también en las ciudades donde su función es disminuir la tensión para el uso comercial o para el uso doméstico, así como el transformador puede ser un reductor también puede ser elevador, este transformador es principalmente utilizado en el principio de la cadena de producción de energía eléctrica, su función es elevar la tensión a cierto punto en el cual permite la transmisión de energía disminuyendo en gran medida sus pérdidas.

\section{Resumen}
Para esta actividad “Diseño y construcción de un transformador eléctrico” de la asignatura máquinas eléctricas I que tiene como finalidad mejorar el student outcome 6 del ABET: “Capacidad para desarrollar y llevar una experimentación adecuada, analizar e interpretar datos y usar juicios de ingeniería para sacar conclusiones.”, se requiere como finalidad construir un modelo de transformador que será evaluado con respecto a una rúbrica, por lo cual, en el siguiente informe se detalla a fondo el diseño del transformador, como lo son sus dimensiones, el calibre de sus bobinas, el número de vueltas de cada bobinado, etc. Así como también se va a especificar los detalles y resultados de tensión, corriente y potencia, los cuales fueron obtenidos por medio de las pruebas de corto circuito y de circuito abierto en los laboratorios de Alta Tensión.
