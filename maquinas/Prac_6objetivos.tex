\section{Objetivos}

\begin{itemize}
    \item Identificar correctamente los bornes de conexión y los devanados internos (de campo y de armadura) de una máquina de corriente continua, con base en su configuración eléctrica y esquema de conexión.
    \item Medir la resistencia de aislamiento de los devanados de la máquina de C.C. utilizando un megóhmetro, verificando su estado eléctrico conforme a los estándares de seguridad eléctrica.
    \item Determinar la caída de tensión en las escobillas durante el funcionamiento de la máquina, evaluando su condición operativa y su influencia en el rendimiento del equipo.
    \item Desarrollar habilidades prácticas en el uso de instrumentos de medición eléctrica aplicados al diagnóstico y análisis de máquinas de corriente continua.
\end{itemize}

\section{Equipos y Materiales}

\begin{itemize}
    \item Máquinas de corriente continua.
    \item Megóhmetro (Megger).
    \item Ohmímetro.
    \item Voltímetro.
    \item Amperímetro.
\end{itemize}

\section{Introducción}

Las máquinas de corriente continua (C.C.) son equipos electromecánicos ampliamente utilizados en aplicaciones donde se requiere un control preciso de la velocidad y el par. Estas máquinas pueden clasificarse, según su tipo de excitación, en tres categorías principales: máquinas en derivación (shunt), en serie y compuestas (compound). Cada una presenta características de operación particulares que las hacen adecuadas para distintas condiciones de carga y aplicaciones industriales.

 Antes de poner en funcionamiento una máquina eléctrica nueva, reparada o que haya estado inactiva, es fundamental realizar una serie de pruebas preliminares que garanticen su correcto funcionamiento y la seguridad durante la operación. Estas pruebas incluyen la identificación y verificación de los bornes de conexión, la medición de la resistencia eléctrica de los devanados, la evaluación de la resistencia de aislamiento, la determinación de la caída de tensión en las escobillas y, en ciertos casos, la comprobación de la zona neutra geométrica. Estas verificaciones permiten prevenir fallos eléctricos, evitar daños en los componentes y asegurar que la máquina cumpla con los estándares técnicos y normativos vigentes.
 Esta práctica de laboratorio tiene como finalidad aplicar estos procedimientos de diagnóstico en una máquina de corriente continua, desarrollando habilidades en el manejo de instrumentos de medición y fortaleciendo la comprensión del comportamiento eléctrico y mecánico de este tipo de máquinas. 