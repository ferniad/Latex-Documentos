\section{Introducción}
La interfaz busca implementar un método numérico mediante App Designer de MATLAB que ayude a identificar aspectos de funcionamiento y emule el comportamiento de un motor de inducción en estado estacionario. Para unas entradas y salidas ya definidas. El alcance del proyecto no busca ser un método más preciso al empleado en la literatura y diversos libros de máquinas eléctricas; al contrario, toma como apoyo dichas ecuaciones ya definidas.

\section{Resumen}
Las máquinas de inducción, particularmente los motores trifásicos, son fundamentales en la industria debido a su robustez, eficiencia y bajo mantenimiento. Para comprender su comportamiento y características operativas, es esencial realizar ensayos experimentales que permitan analizar su rendimiento en distintas condiciones. Entre las pruebas más comunes se encuentran la prueba de vacío y la prueba de rotor bloqueado, las cuales proporcionan información clave para la obtención de los parámetros del equivalente eléctrico del motor. Para la creación del codigo ya se parte de un equivalente de dichas pruebas realizadas y del modelo del circiuto se tienen las entradas y a partir de estas entradas hallar las salidas se tomó un modelo de motor ya definido en la literatura —con su significado físico—. Se concatenaron las ecuaciones del modelo en función de las entradas y salidas definidas.

