\section*{Conclusiones}

\begin{itemize}
    \item Las pruebas de vacío, corto circuito, rendimiento y regulación son esenciales para la caracterización y evaluación de autotransformadores. La prueba de vacío evalúa las pérdidas de núcleo, mientras que la prueba de corto circuito permite medir las pérdidas de cobre y la impedancia de los devanados bajo carga. Por último, la prueba de rendimiento proporciona información sobre la eficiencia global del transformador cuando está operando con carga. La combinación de estas pruebas proporciona una imagen integral del comportamiento del transformador, lo cual es crucial para garantizar su rendimiento y eficiencia a lo largo de su vida útil.
    \item A lo largo del laboratorio, se estudió el comportamiento de un transformador monofásico operando como autotransformador, evaluando su rendimiento y regulación bajo diferentes condiciones de carga, las cuales tienen una gran influencia en su rendimiento depediendo el tipo de carga, si es una carga RC conectada en paralelo tiende a aunmentar su tension de salida en comparacion con una carga netamente resistiva, mientras que con una carga inductiva conectada en serie su nivel de tension de salida disminuye, pero si conectamos una carga RLC está tiene un comportamiento de compensacion, si las cargas inductivas (L) y capacitivas (C) tienen valores similares.
    \end{itemize}

