\section*{Linealización Polinomial de la Corriente de Línea}

A partir de los datos de corriente de excitación ($I_{\text{excitación}}$) y corriente de línea ($I_{\text{línea}}$) de la conexión en paralelo de los motores síncronos, se linealizaron a un polinomio de orden 4 y se obtuvo la siguiente gráfica.

La ecuación polinomial resultante es:
\begin{multline*}
I_{\text{línea}} = -60.3563 + 218.4110 \cdot I_{\text{excitación}} \\ 
- 279.3314 \cdot I_{\text{excitación}}^2 + 153.6118 \cdot I_{\text{excitación}}^3 \\ 
- 30.6968 \cdot I_{\text{excitación}}^4
\end{multline*}

\subsection*{Datos Experimentales}

Los datos utilizados para la linealización se presentan en la \textbf{Tabla \ref{tab:datos_motores}}:

\begin{table}[H]
    \centering
    \caption{Datos de Corriente de Excitación y Corriente de Línea.}
    \label{tab:datos_motores}
    \begin{tabular}{cc}
        \toprule
        \textbf{$I_{\text{excitación}}$} (A) & \textbf{$I_{\text{línea}}$} (A) \\
        \midrule
        0.90 & 1.80 \\
        1.05 & 1.50 \\
        1.10 & 1.45 \\
        1.30 & 1.30 \\
        1.40 & 1.50 \\
        1.40 & 1.55 \\
        1.45 & 1.65 \\
        1.50 & 1.80 \\
        \bottomrule
    \end{tabular}
\end{table}

\subsection*{Gráficas}

\begin{figure}[H]
    \centering
    \includegraphics[width=\linewidth]{figs/puntos recta.png}
    \caption{Puntos de datos experimentales.}
    \label{fig:puntos_recta}
\end{figure}

\begin{figure}[H]
    \centering
    \includegraphics[width=\linewidth]{figs/puntos curva.png}
    \caption{Polinomio de orden 4 resultante de la linealización de los datos.}
    \label{fig:puntos_curva}
\end{figure}