
\documentclass[conference]{IEEEtran}

% --- Idioma y Codificación (Se cargan primero) ---
\usepackage[utf8]{inputenc}
\usepackage[spanish]{babel}

% --- Matemáticas, Formato y Estructuras Avanzadas ---
% Paquetes matemáticos y de formato avanzado agrupados:
\usepackage{amssymb, amsmath, amsthm, empheq, bm}
% Paquetes para estructuras enmarcadas y colores específicos:
\usepackage{mdframed, float}
\usepackage{color, colortbl, xcolor}
% --- Figuras, Tablas y Gráficos ---
% Paquetes para imágenes, tablas y gráficos agrupados:
\usepackage{graphicx, subcaption, caption, booktabs, multirow, psfrag}
\usepackage{pgfplots} % Se mantiene aparte para mejor visibilidad
% --- Hipervínculos y Documentación ---
\usepackage{hyperref, bookmark}

%%%%%%%%%%%%%%%%%%%%%%%%%%%%%%%%%%%%%%%%%%%%%%%%%%%%%%%%%%%%%%%%%%%%%%%%
% --- Paquete para Código Fuente ---}
% Paquete de colores (ya lo tienes, solo aseguramos que está antes de la config. de listings)


% --- Configuración de Estilo de MATLAB ---
%%%%%%%%%%%%%%%%%%%%%%%%%%%%%%%%%%%%%%%%%%%%%%%%%%%%%%%%%%%%%%%%%%%%%%%%
 % Para tablas con ancho de columna flexible
 % Para bloques de código
 % Para ajustar márgenes si es necesario
% --- AJUSTE CRÍTICO: La clase IEEEtran ya maneja los márgenes.
% **Se recomienda ENCARECIDAMENTE comentar o eliminar esta línea**
% para mantener el formato oficial de la conferencia.
% \usepackage{geometry}
% \geometry{a4paper, margin=1in}

%%%%%%%%%%%%%%%%%%%%%%%%%%%%%%%%%%%%%%%%%%%%%%%%%%%%%%%%%%%%%%%%%%%%%%%%%%%%%%%

% Other Settings

%%%%%%%%%%%%%%%%%%%%%%%%%% Define some useful colors %%%%%%%%%%%%%%%%%%%%%%%%%%
% Definición de colores (mantener separados para claridad en la configuración)
\definecolor{lightgreen}{HTML}{A9D18E}
\definecolor{lightred}{HTML}{F4C7C3}
\definecolor{ocre}{RGB}{243,102,25}
\definecolor{mygray}{RGB}{243,243,244}
\definecolor{deepGreen}{RGB}{26,111,0}
\definecolor{shallowGreen}{RGB}{235,255,255}
\definecolor{deepBlue}{RGB}{61,124,222}
\definecolor{shallowBlue}{RGB}{235,249,255}
%%%%%%%%%%%%%%%%%%%%%%%%%%%%%%%%%%%%%%%%%%%%%%%%%%%%%%%%%%%%%%%%%%%%%%%%%%%%%%%

%%%%%%%%%%%%%%%%%%%%%%%%%% Define an orangebox command %%%%%%%%%%%%%%%%%%%%%%%%
\newcommand\orangebox[1]{\fcolorbox{ocre}{mygray}{\hspace{1em}#1\hspace{1em}}}
%%%%%%%%%%%%%%%%%%%%%%%%%%%%%%%%%%%%%%%%%%%%%%%%%%%%%%%%%%%%%%%%%%%%%%%%%%%%%%%

%%%%%%%%%%%%%%%%%%%%%%%%%%%% Spanish Environments %%%%%%%%%%%%%%%%%%%%%%%%%%%%%
% Definición de estilos y entornos
\newtheoremstyle{mytheoremstyle}{3pt}{3pt}{\normalfont}{0cm}{\rmfamily\bfseries}{}{1em}{{\color{black}\thmname{#1}~\thmnumber{#2}}\thmnote{\,--\,#3}}
\newtheoremstyle{myproblemstyle}{3pt}{3pt}{\normalfont}{0cm}{\rmfamily\bfseries}{}{1em}{{\color{black}\thmname{#1}~\thmnumber{#2}}\thmnote{\,--\,#3}}
\theoremstyle{mytheoremstyle}
\newmdtheoremenv[linewidth=1pt,backgroundcolor=shallowGreen,linecolor=deepGreen,leftmargin=0pt,innerleftmargin=20pt,innerrightmargin=20pt,]{theorem}{Theorem}[section]
\theoremstyle{mytheoremstyle}
\newmdtheoremenv[linewidth=1pt,backgroundcolor=shallowBlue,linecolor=deepBlue,leftmargin=0pt,innerleftmargin=20pt,innerrightmargin=20pt,]{definition}{Definition}[section]
\theoremstyle{myproblemstyle}
\newmdtheoremenv[linecolor=black,leftmargin=0pt,innerleftmargin=10pt,innerrightmargin=10pt,]{problem}{Problem}[section]
%%%%%%%%%%%%%%%%%%%%%%%%%%%%%%%%%%%%%%%%%%%%%%%%%%%%%%%%%%%%%%%%%%%%%%%%%%%%%%%

%%%%%%%%%%%%%%%%%%%%%%%%%%%%%%% Plotting Settings %%%%%%%%%%%%%%%%%%%%%%%%%%%%%
\usepgfplotslibrary{colorbrewer}
\pgfplotsset{width=8cm,compat=1.9}
%%%%%%%%%%%%%%%%%%%%%%%%%%%%%%%%%%%%%%%%%%%%%%%%%%%%%%%%%%%%%%%%%%%%%%%%%%%%%%%

%%%%%%%%%%%%%%%%%%%%%%%%%%%%%%% Title & Author %%%%%%%%%%%%%%%%%%%%%%%%%%%%%%%%
\author{\IEEEauthorblockN{Jose David Hernández Rodriguez, Juan Andrés Díaz López, David Nicolas Ortega Peña,\\ Daniel Fernando Aranda Contreras}
\IEEEauthorblockA{Escuela E3T, Universidad Industrial de Santander\\
Correo electrónico: \{jose2221117, juan2205102, david2225138F1883, daniel2221648\}@correo.uis.edu.co}}
%%%%%%%%%%%%%%%%%%%%%%%%%%%%%%%%%%%%%%%%%%%%%%%%%%%%%%%%%%%%%%%%%%%%%%%%%%%%%%%

\begin{document}
    % Título
    \title{APLICACIÓN DEL CÓDIGO DE MEDIDA}
    \maketitle

    %\begin{abstract}
    
    %\end{abstract}

    \begin{IEEEkeywords}
        si-prueba.
    \end{IEEEkeywords}

    \begin{abstract}
        si.
    \end{abstract}


\begin{table}[h]
    \centering
    \caption{Tensión de línea en conexión delta ($\Delta$).}
    \label{tab:delta_tension_corregida}
    \begin{tabular}{|c|c|}
        \hline
        \textbf{Terminales} & \textbf{Tensión de Línea ($V_L$) [V]} \\
        \hline
        W2-U2 & 50.48 \\
        \hline
        W2-V2 & 50.55 \\
        \hline
        U2-V2 & 50.42 \\
        \hline
        \textbf{Relación Teórica} & $V_L = V_{\text{devanado}}$ \\
        \hline
    \end{tabular}
\end{table}
 
\begin{table}[h]
    \centering
    \caption{Tensión de línea en conexión Y.}
    \label{tab:y_tension_mod}
    \begin{tabular}{|c|c|c|}
        \hline
        & \textbf{Tensión de Línea ($V_L$) [V]} & \textbf{Tensión de Fase ($V_F$) [V]} \\
        \hline
        W2-U2 & 87.35 & 50.44 \\
        \hline
        W2-V2 & 87.20 & 50.35 \\
        \hline
        U2-V2 & 87.45 & 50.50 \\
        \hline
        \textbf{Relación Teórica} & \multicolumn{2}{|c|}{$V_F = \frac{V_L}{\sqrt{3}}$} \\
        \hline
    \end{tabular}
\end{table}


\begin{itemize}
    \item Para el caso de la conexión en delta ($\Delta$), se observa que la tensión de línea medida en los terminales W2-U2, W2-V2 y U2-V2 (ver Tabla \ref{tab:delta_tension_corregida}). Estas mediciones están en concordancia con la fuente balanceada.
    \item Para el caso de la conexión en Y, se observa que la tensión de línea medida en los terminales W2-U2, W2-V2 y U2-V2 (ver Tabla \ref{tab:y_tension_mod}). Estas mediciones están en concordancia con la fuente balanceada.
 
\end{itemize}

En ambas configuraciones, las mediciones de tensión de línea están en concordancia con las relaciones teóricas esperadas, confirmando la correcta implementación de las conexiones delta y Y en el sistema trifásico y que el generador mantiene establidad con los valores esperados.



\end{document}