%%%%%%%%%%%%%%%%%%%%%%%%%%%%% Define Article %%%%%%%%%%%%%%%%%%%%%%%%%%%%%%%%%%
\documentclass[conference]{IEEEtran}
%%%%%%%%%%%%%%%%%%%%%%%%%%%%%%%%%%%%%%%%%%%%%%%%%%%%%%%%%%%%%%%%%%%%%%%%%%%%%%%

%%%%%%%%%%%%%%%%%%%%%%%%%%%%% Using Packages %%%%%%%%%%%%%%%%%%%%%%%%%%%%%%%%%%
\usepackage{geometry}
\usepackage{graphicx}
\usepackage{amssymb}
\usepackage{amsmath}
\usepackage{amsthm}

\usepackage{empheq}
\usepackage{mdframed}
\usepackage{booktabs}
\usepackage{lipsum}
\usepackage{graphicx}
\usepackage{color}
\usepackage{psfrag}
\usepackage{pgfplots}
\usepackage{bm}
\usepackage[es-uppernames]{babel} % Soporte para el idioma inglés
\usepackage[utf8]{inputenc} % Codificación UTF-8
\usepackage{amsmath}        % Soporte para ecuaciones matemáticas
\usepackage{graphicx}       % Manejo de imágenes
\usepackage{hyperref}       % Hipervínculos
\usepackage{caption}        % Formato para figuras
\usepackage{multirow}
\usepackage{subcaption}
\usepackage{biblatex}
\usepackage{csquotes}
\usepackage{bookmark}
%%%%%%%%%%%%%%%%%%%%%%%%%%%%%%%%%%%%%%%%%%%%%%%%%%%%%%%%%%%%%%%%%%%%%%%%%%%%%%%

% Other Settings

%%%%%%%%%%%%%%%%%%%%%%%%%% Page Setting %%%%%%%%%%%%%%%%%%%%%%%%%%%%%%%%%%%%%%%
\geometry{letterpaper, margin=1in}

%%%%%%%%%%%%%%%%%%%%%%%%%% Define some useful colors %%%%%%%%%%%%%%%%%%%%%%%%%%
\definecolor{ocre}{RGB}{243,102,25}
\definecolor{mygray}{RGB}{243,243,244}
\definecolor{deepGreen}{RGB}{26,111,0}
\definecolor{shallowGreen}{RGB}{235,255,255}
\definecolor{deepBlue}{RGB}{61,124,222}
\definecolor{shallowBlue}{RGB}{235,249,255}
%%%%%%%%%%%%%%%%%%%%%%%%%%%%%%%%%%%%%%%%%%%%%%%%%%%%%%%%%%%%%%%%%%%%%%%%%%%%%%%

%%%%%%%%%%%%%%%%%%%%%%%%%% Define an orangebox command %%%%%%%%%%%%%%%%%%%%%%%%
\newcommand\orangebox[1]{\fcolorbox{ocre}{mygray}{\hspace{1em}#1\hspace{1em}}}
%%%%%%%%%%%%%%%%%%%%%%%%%%%%%%%%%%%%%%%%%%%%%%%%%%%%%%%%%%%%%%%%%%%%%%%%%%%%%%%

%%%%%%%%%%%%%%%%%%%%%%%%%%%% English Environments %%%%%%%%%%%%%%%%%%%%%%%%%%%%%
\newtheoremstyle{mytheoremstyle}{3pt}{3pt}{\normalfont}{0cm}{\rmfamily\bfseries}{}{1em}{{\color{black}\thmname{#1}~\thmnumber{#2}}\thmnote{\,--\,#3}}
\newtheoremstyle{myproblemstyle}{3pt}{3pt}{\normalfont}{0cm}{\rmfamily\bfseries}{}{1em}{{\color{black}\thmname{#1}~\thmnumber{#2}}\thmnote{\,--\,#3}}
\theoremstyle{mytheoremstyle}
\newmdtheoremenv[linewidth=1pt,backgroundcolor=shallowGreen,linecolor=deepGreen,leftmargin=0pt,innerleftmargin=20pt,innerrightmargin=20pt,]{theorem}{Theorem}[section]
\theoremstyle{mytheoremstyle}
\newmdtheoremenv[linewidth=1pt,backgroundcolor=shallowBlue,linecolor=deepBlue,leftmargin=0pt,innerleftmargin=20pt,innerrightmargin=20pt,]{definition}{Definition}[section]
\theoremstyle{myproblemstyle}
\newmdtheoremenv[linecolor=black,leftmargin=0pt,innerleftmargin=10pt,innerrightmargin=10pt,]{problem}{Problem}[section]
%%%%%%%%%%%%%%%%%%%%%%%%%%%%%%%%%%%%%%%%%%%%%%%%%%%%%%%%%%%%%%%%%%%%%%%%%%%%%%%

%%%%%%%%%%%%%%%%%%%%%%%%%%%%%%% Plotting Settings %%%%%%%%%%%%%%%%%%%%%%%%%%%%%
\usepgfplotslibrary{colorbrewer}
\pgfplotsset{width=8cm,compat=1.9}
%%%%%%%%%%%%%%%%%%%%%%%%%%%%%%%%%%%%%%%%%%%%%%%%%%%%%%%%%%%%%%%%%%%%%%%%%%%%%%%

%%%%%%%%%%%%%%%%%%%%%%%%%%%%%%% Title & Author %%%%%%%%%%%%%%%%%%%%%%%%%%%%%%%%
\author{\IEEEauthorblockN{Brayan Joanne Ballesteros Meza, Daniel Fernando Aranda Contreras, Jonathan Stiven Murcia Suarez}
\IEEEauthorblockA{Escuela E3T, Universidad Industrial de Santander\\
Correo electrónico: \{brayan2222069, daniel2221648, jonathan2225092\}@correo.uis.edu.co}}


%%%%%%%%%%%%%%%%%%%%%%%%%%%%%%%%%%%%%%%%%%%%%%%%%%%%%%%%%%%%%%%%%%%%%%%%%%%%%%%

\begin{document}

% Título
\title{Impacto de la selección de la frecuencia de muestreo en la estimación de parámetros de un sistema eléctrico.}

\maketitle
% Resumen

% Palabras clave        
\begin{IEEEkeywords}
    Frecuencia de muestreo, Mediciones eléctricas, Análisis de señales, Valor RMS, Muestreo de señales, Errores de estimación, Parámetros del sistema eléctrico, Comparación de procesos de muestreo.   
\end{IEEEkeywords}

\section*{Resultados de Medición por Frecuencia}

\subsection*{Tabla T2: Resultados para 180 Hz}
\begin{table}[h!]
    \centering
    \begin{tabular}{@{}ccccc@{}}
        \toprule
        Vrms [Vrms] & Irms [Arms] & P [W] & Q [VAR] & S [VA] \\ \midrule
        110 & 11.0293 & 1155.1 & 371.1589 & 1213.2 \\ 
        \bottomrule
    \end{tabular}
    \caption{Resultados medidos a 180 Hz.}
\end{table}

\subsection*{Tabla T2: Resultados para 200 Hz}
\begin{table}[h!]
    \centering
    \begin{tabular}{@{}ccccc@{}}
        \toprule
        Vrms & Irms & P & Q & S \\ \midrule
        118.8136 & 11.7527 & 1.1551e+03 & 784.6681 & 1.3964e+03 \\ 
        \bottomrule
    \end{tabular}
    \caption{Resultados medidos a 200 Hz.}
\end{table}

\subsection*{Tabla T3: Resultados para 240 Hz}
\begin{table}[h!]
    \centering
    \begin{tabular}{@{}ccccc@{}}
        \toprule
        Vrms & Irms & P & Q & S \\ \midrule
        110.0000 & 11.0293 & 1.1551e+03 & 371.1590 & 1.2132e+03 \\ 
        \bottomrule
    \end{tabular}
    \caption{Resultados medidos a 240 Hz.}
\end{table}

\subsection*{Tabla T4: Resultados para 280 Hz}
\begin{table}[h!]
    \centering
    \begin{tabular}{@{}ccccc@{}}
        \toprule
        Vrms & Irms & P & Q & S \\ \midrule
        116.6726 & 11.5761 & 1.1551e+03 & 699.9905 & 1.3506e+03 \\ 
        \bottomrule
    \end{tabular}
    \caption{Resultados medidos a 280 Hz.}
\end{table}

\subsection*{Tabla Final: Resultados para 100 Hz}
\begin{table}[h!]
    \centering
    \begin{tabular}{@{}ccccc@{}}
        \toprule
        Vrms & Irms & P & Q & S \\ \midrule
        118.8136 & 13.2088 & 1.1551e+03 & 1.0625e+03 & 1.5694e+03 \\ 
        \bottomrule
    \end{tabular}
    \caption{Resultados medidos a 100 Hz.}
\end{table}



\end{document}