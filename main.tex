
%%%%%%%%%%%%%%%%%%%%%%%%%%%%% Define Article %%%%%%%%%%%%%%%%%%%%%%%%%%%%%%%%%%
\documentclass[conference]{IEEEtran}
%%%%%%%%%%%%%%%%%%%%%%%%%%%%%%%%%%%%%%%%%%%%%%%%%%%%%%%%%%%%%%%%%%%%%%%%%%%%%%%

%%%%%%%%%%%%%%%%%%%%%%%%%%%%% Using Packages %%%%%%%%%%%%%%%%%%%%%%%%%%%%%%%%%%
\usepackage{geometry}
\usepackage{graphicx}
\usepackage{amssymb}
\usepackage{amsmath}
\usepackage{amsthm}

\usepackage{empheq}
\usepackage{mdframed}
\usepackage{booktabs}
\usepackage{lipsum}
\usepackage{graphicx}
\usepackage{color}
\usepackage{psfrag}
\usepackage{pgfplots}
\usepackage{bm}
\usepackage[spanish]{babel}
\usepackage[utf8]{inputenc} % Codificación UTF-8
\usepackage{amsmath}        % Soporte para ecuaciones matemáticas
\usepackage{graphicx}       % Manejo de imágenes
\usepackage{hyperref}       % Hipervínculos
\usepackage{caption}        % Formato para figuras
\usepackage{multirow}
\usepackage{subcaption}
\usepackage{biblatex}
\usepackage{csquotes}
\usepackage{bookmark}
%%%%%%%%%%%%%%%%%%%%%%%%%%%%%%%%%%%%%%%%%%%%%%%%%%%%%%%%%%%%%%%%%%%%%%%%%%%%%%%

% Other Settings

%%%%%%%%%%%%%%%%%%%%%%%%%% Page Setting %%%%%%%%%%%%%%%%%%%%%%%%%%%%%%%%%%%%%%%
\geometry{a4paper, margin=1in}

%%%%%%%%%%%%%%%%%%%%%%%%%% Define some useful colors %%%%%%%%%%%%%%%%%%%%%%%%%%
\definecolor{ocre}{RGB}{243,102,25}
\definecolor{mygray}{RGB}{243,243,244}
\definecolor{deepGreen}{RGB}{26,111,0}
\definecolor{shallowGreen}{RGB}{235,255,255}
\definecolor{deepBlue}{RGB}{61,124,222}
\definecolor{shallowBlue}{RGB}{235,249,255}
%%%%%%%%%%%%%%%%%%%%%%%%%%%%%%%%%%%%%%%%%%%%%%%%%%%%%%%%%%%%%%%%%%%%%%%%%%%%%%%

%%%%%%%%%%%%%%%%%%%%%%%%%% Define an orangebox command %%%%%%%%%%%%%%%%%%%%%%%%
\newcommand\orangebox[1]{\fcolorbox{ocre}{mygray}{\hspace{1em}#1\hspace{1em}}}
%%%%%%%%%%%%%%%%%%%%%%%%%%%%%%%%%%%%%%%%%%%%%%%%%%%%%%%%%%%%%%%%%%%%%%%%%%%%%%%

%%%%%%%%%%%%%%%%%%%%%%%%%%%% English Environments %%%%%%%%%%%%%%%%%%%%%%%%%%%%%
\newtheoremstyle{mytheoremstyle}{3pt}{3pt}{\normalfont}{0cm}{\rmfamily\bfseries}{}{1em}{{\color{black}\thmname{#1}~\thmnumber{#2}}\thmnote{\,--\,#3}}
\newtheoremstyle{myproblemstyle}{3pt}{3pt}{\normalfont}{0cm}{\rmfamily\bfseries}{}{1em}{{\color{black}\thmname{#1}~\thmnumber{#2}}\thmnote{\,--\,#3}}
\theoremstyle{mytheoremstyle}
\newmdtheoremenv[linewidth=1pt,backgroundcolor=shallowGreen,linecolor=deepGreen,leftmargin=0pt,innerleftmargin=20pt,innerrightmargin=20pt,]{theorem}{Theorem}[section]
\theoremstyle{mytheoremstyle}
\newmdtheoremenv[linewidth=1pt,backgroundcolor=shallowBlue,linecolor=deepBlue,leftmargin=0pt,innerleftmargin=20pt,innerrightmargin=20pt,]{definition}{Definition}[section]
\theoremstyle{myproblemstyle}
\newmdtheoremenv[linecolor=black,leftmargin=0pt,innerleftmargin=10pt,innerrightmargin=10pt,]{problem}{Problem}[section]
%%%%%%%%%%%%%%%%%%%%%%%%%%%%%%%%%%%%%%%%%%%%%%%%%%%%%%%%%%%%%%%%%%%%%%%%%%%%%%%

%%%%%%%%%%%%%%%%%%%%%%%%%%%%%%% Plotting Settings %%%%%%%%%%%%%%%%%%%%%%%%%%%%%
\usepgfplotslibrary{colorbrewer}
\pgfplotsset{width=8cm,compat=1.9}
%%%%%%%%%%%%%%%%%%%%%%%%%%%%%%%%%%%%%%%%%%%%%%%%%%%%%%%%%%%%%%%%%%%%%%%%%%%%%%%

%%%%%%%%%%%%%%%%%%%%%%%%%%%%%%% Title & Author %%%%%%%%%%%%%%%%%%%%%%%%%%%%%%%%
\author{\IEEEauthorblockN{Brayan Joanne Ballesteros Meza, Brayhan Steven Delgado Rueda, Daniel Fernando Aranda Contreras,\\ Jonathan Stiven Murcia Suarez}
\IEEEauthorblockA{Escuela E3T, Universidad Industrial de Santander\\
Correo electrónico: \{brayan2222069, brayan2212088, daniel2221648, jonathan2225092\}@correo.uis.edu.co}}


%%%%%%%%%%%%%%%%%%%%%%%%%%%%%%%%%%%%%%%%%%%%%%%%%%%%%%%%%%%%%%%%%%%%%%%%%%%%%%%

\begin{document}
    % Título
    \title{Impacto de la selección de la frecuencia de muestreo en la estimación de parámetros de un sistema eléctrico.}
    \maketitle
    % Resumen
    % Palabras clave        
    \begin{IEEEkeywords}
        Frecuencia de muestreo, Mediciones eléctricas, Análisis de señales, Valor RMS, Muestreo de señales, Errores de estimación, Parámetros del sistema eléctrico, Comparación de procesos de muestreo.   
    \end{IEEEkeywords}

    \section*{Resultados de Medición por Frecuencia Para el Panel de Distrubución} 
    
\subsection*{Tabla T2: Resultados para 180 Hz}
    Para este caso de frecuencia de muestreo se consiguen valores con un porcentaje de error del 0.0001\% con relación a los datos obtenidos analíticamente, se aprecia que la frecuencia a la que se esta muestreando es 3 veces el valor de la frecuencia fundamental por lo cual no se incumple el teorema de Nyquist-Shannon.
    \begin{table}[h!]
        \centering
        \begin{tabular}{@{}ccccc@{}}
            \toprule
            Vrms [Vrms] & Irms [Arms] & P [W] & Q [VAR] & S [VA] \\ \midrule
                110 & 11.0293 & 1155.1 & 371.1589 & 1213.2\\ 
                \bottomrule
        \end{tabular}
        \caption{Resultados medidos a 180 Hz.}
    \end{table}



    \subsection*{Tabla T2: Resultados para 200 Hz}
    Para este caso es posible conseguir valores mas precisos, se requería de tres ventanas de observación de las cuales solo se están usando dos para tomar medidas y además de eso diez medidas de las cuales solo tenemos información de seis en esas dos ventanas de observación.
    \begin{table}[h!]
        \centering
        \begin{tabular}{@{}ccccc@{}}
            \toprule
            Vrms [Vrms] & Irms [Arms] & P [W] & Q [VAR] & S [VA] \\ \midrule
            118.8136 & 11.7527 & 1347.6 & 365.9631 & 1396.4 \\ 
            \bottomrule
        \end{tabular}
        \caption{Resultados medidos a 200 Hz.}
    \end{table}

    \subsection*{Tabla T3: Resultados para 240 Hz}
    Para este caso de frecuencia de muestreo se consiguen valores con un porcentaje de error del 0.0001\% exactamente igual a la frecuencia de 180 [Hz]. Con relación a los datos obtenidos analíticamente, se aprecia que la frecuencia a la que se esta muestreando es 4 veces el valor de la frecuencia fundamental por lo cual no se incumple el teorema de Nyquist-Shannon. 

    \begin{table}[h!]
        \centering
        \begin{tabular}{@{}ccccc@{}}
            \toprule
            Vrms [Vrms] & Irms [Arms] & P [W] & Q [VAR] & S [VA] \\ \midrule
            110.0000 & 11.0293 & 1155.1 & 371.1590 & 1213.2 \\ 
            \bottomrule
        \end{tabular}
        \caption{Resultados medidos a 240 Hz.}
    \end{table}

    \subsection*{Tabla T4: Resultados para 280 Hz}
    Similar al caso de la frecuencia de 200 Hz para este se requería de catorce muestras de las cuales solo se tienen nueve, además de eso se requería de tres ventanas de observación de las cuales solo se tienen dos.
    \begin{table}[h!]
        \centering
        \begin{tabular}{@{}ccccc@{}}
            \toprule
            Vrms [Vrms] & Irms [Arms] & P [W] & Q [VAR] & S [VA] \\ \midrule
            116.6726 & 11.5761 & 1299.4 & 368.2449 & 1350.6  \\ 
            \bottomrule
        \end{tabular}
        \caption{Resultados medidos a 280 Hz.}
    \end{table}

    \subsection*{Tabla Final: Resultados para 100 Hz}
    Finalmente en este caso se da antialiasing, puesto que la frecuencia de muestreo de la señal es menor al doble de la frecuencia fundamental de la red.
    \begin{table}[h!]
        \centering
        \begin{tabular}{@{}ccccc@{}}
            \toprule
            Vrms [Vrms] & Irms [Arms] & P [W] & Q [VAR] & S [VA] \\ \midrule
            118.8136 & 13.2088 & 1537.9 & 312.5425 & 1569.4 \\ 
            \bottomrule
        \end{tabular}
        \caption{Resultados medidos a 100 Hz.}
    \end{table}
	
    
\section*{Factor de Potencia (FP) y Potencia Activa Consumida por el Parlante}

\begin{table}[h!]
\centering
\begin{tabular}{ccc}
\toprule
\textbf{Frecuencia [Hz]} & \textbf{FP (Factor de Potencia)} & \textbf{($P_{\text{parlante}}$) [W]} \\
\midrule
100  & 0.7361 & 1500.9 \\
180  & 0.9521 & 1129.2 \\
200  & 0.8272 & 1318.2 \\
240  & 0.9521 & 1129.2 \\
280  & 0.8552 & 1271.0 \\
\bottomrule
\end{tabular}
\caption{Resultados de FP y potencia activa consumida por el parlante para diferentes frecuencias.}
\label{table:resultados}
\end{table}

    
\end{document}  
