

\section{Criterios para el trazado de la ruta}

\subsection{Criterios Socioeconómicos}
\begin{itemize}
    \item Población y Demografía
    \subitem Se consideran la distribución y densidad poblacional, así como la estructura de edad y género en las zonas de influencia. Estos datos permiten dimensionar la demanda de servicios, la disponibilidad de mano de obra y el posible impacto social.
    \\
    \item Tenencia de la Tierra y Restitución
    \subitem Se analiza la distribución de la propiedad, los tipos de tenencia (privada, colectiva, pública) y la existencia de procesos de restitución de tierras. Esto permite anticipar posibles conflictos de uso y definir estrategias para la gestión predial.
    \\
    \item Comunidades Étnicas y Actores Sociales
    \subitem Se identifican diferentes comunidades. El objetivo es garantizar la participación, el reconocimiento de sus derechos y la consideración de sus dinámicas culturales.
    \\
    \item Superposición con Otros Proyectos
    \subitem Se consideran iniciativas en curso o planificadas (infraestructura, hidrocarburos, minería, agricultura, etc.) para identificar sinergias y prevenir impactos acumulativos o conflictos en la ocupación del territorio.
\end{itemize}

\subsection{Criterios Físicos}
\begin{itemize}
    \item Estabilidad del Terreno
    \subitem Se evalúan las características geológicas y estructurales para determinar posibles deformaciones en la corteza y la historia geológica del área. Asimismo, se consideran amenazas naturales como la sismicidad y la remoción en masa (deslizamientos, derrumbes, etc.), que pueden afectar la infraestructura y la seguridad en la zona de influencia.
    \\
    \item Contaminación y Erosión
    \subitem Se identifican los factores que podrían generar deterioro en la calidad del aire, del suelo y de las fuentes hídricas. De igual manera, se revisan los procesos de erosión y sedimentación que pueden incrementarse con las actividades de construcción, buscando medidas de control y prevención.
    \\
    \item Recursos Hídricos
    \subitem Incluye la revisión de la disponibilidad y calidad del agua, tanto superficial como subterránea. Se procura salvaguardar los cuerpos de agua y sus zonas de protección, asegurando la sostenibilidad del recurso y minimizando posibles impactos en los ecosistemas asociados.
    \\
    \item Uso del Suelo y Paisaje
    \subitem Se analizan las vocaciones y usos actuales del suelo, así como los lineamientos de ordenamiento territorial. El objetivo es integrar la infraestructura de manera armónica con el entorno, reduciendo impactos visuales y protegiendo la calidad del paisaje local.
\end{itemize}
%Criterios rabon Dai
\subsection{Criterios de seguridad}
\begin{itemize}
    \item El trazado debe mantenerse alejado de áreas con amenaza de deslizamientos, inundaciones, fallas geológicas o incendios forestales, frecuentes en algunas zonas del Huila.
    
    \item La ruta debe permitir el acceso seguro para brigadas de mantenimiento y reparación sin exponer al personal a riesgos innecesarios.
    
    \item Las líneas de transmisión deben diseñarse y ubicarse considerando la protección con zonas habitadas, vías, cultivos, estructuras y el entorno, manteniendo distancias mínimas de seguridad que cumplan con las normativas eléctricas vigentes para minimizar riesgos de descargas, cortocircuitos o accidentes.
\end{itemize}

\subsection{Criterios bióticos}
\begin{itemize}
    \item Ecosistemas y plan de ordenación forestal (POF)
    \subitem - Se identifican zonas con alta sensibilidad biótica, como áreas de protección y restauración de bosques. El POF pone énfasis en la importancia de proteger recursos naturales y la estructura ecosistémica. 
    \subitem - Parte del área de estudio se clasifica bajo categorías de alta sensibilidad, principalmente aquellas que requieren conservación o restauración.
    \\
    \item Sensibilidad del medio biótico
    \subitem - La sensibilidad biótica del área de estudio se clasifica en varias categorías (muy alta, alta, moderada y baja). Se observan ecosistemas naturales, como bosques y humedales, que tienen una sensibilidad muy alta, mientras que áreas artificializadas presentan menor sensibilidad.
    \subitem - La distribución de esta sensibilidad es clave para la gestión ambiental y la preservación de los recursos.
    \\
    \item Variables Relevantes
    \subitem - Se consideran variables que impactan la calidad y conservación del medio biótico, incluyendo ecosistemas estratégicos, áreas protegidas, y registros de ecosistemas. 
    \subitem - Las variables se utilizan para realizar análisis de sensibilidad y determinar las áreas prioritarias para la conservación.
    \\
    \item Zonificación del medio biótico
    \subitem - Se ha realizado una zonificación que asigna grados de sensibilidad a diferentes áreas (muy alta, alta, moderada y baja). 
    \subitem - La zonificación se basa en la evaluación de las características ecológicas y la necesidad de protección, definida por el área total y el porcentaje que cada categoría representa dentro del territorio estudiado.



        \begin{figure}{h!}
            \centering
            \includegraphics[width=0.5\textwidth]{1mer avance foticos/Biotico.png}
            \caption{Imagen tomada de ANÁLISIS ÁREA DE ESTUDIO PRELIMINAR Y ALERTAS TEMPRANAS. PROYECTO NUEVA SUBESTACIÓN HUILA 230 KV Y LÍNEAS DE TRANSMISIÓN ASOCIADAS. UPME N° 01 - 2022..} % Título de la figura
            \label{fig:Torre-Dibujo} % Etiqueta para referencias
        \end{figure}



\end{itemize}




\subsection*{Esquema unifilar y diagrama esquemático}
A continuación se presenta el digrama unifilar y esquemático del proyecto que permite identificar la conexión de las lineas de transmisión que conecta conecta la nueva Subestación Huila con las asociadas.
\begin{figure}[h!] % 'h' coloca la figura aquí
    \centering % Centra la imagen
    \begin{subfigure}{0.5\textwidth}
        \includegraphics[width=1\textwidth]{1mer avance foticos/Esquema unifilar diagrama esquemático.png}
        \caption{en la figura del diagrama unifilar esquemático se logra evidenciar de una manera mas sencilla como ira conectada la nueva subestación Huila.} % Título de la figura
        \label{fig:Esquema} % Etiqueta para referencias
    \end{subfigure}
    \hfill % Espacio horizontal entre las subfiguras
    \begin{subfigure}{0.5\textwidth}
        \centering % Centra la imagen
        \includegraphics[width=1\textwidth]{1mer avance foticos/Esquema unifilar de la subestación huila 230kv.png}
        \caption{Esquema unifilar de la subestación huila 230kv.} % Título de la figura
        \label{fig:unifilar} % Etiqueta para referencias
    \end{subfigure}
    \label{fig:dos-imagenes}
\end{figure}

