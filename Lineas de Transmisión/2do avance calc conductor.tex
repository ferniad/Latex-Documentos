\section{Elección del conductor: Flint (AAAC)}

En el diseño de la línea de transmisión asociada a la nueva subestación Huila 230 kV, se consideraron diferentes tipos de conductores con base en criterios eléctricos, mecánicos y ambientales. Las opciones evaluadas fueron AAC (Aluminium Conductor, All), ACAR (Aluminium Conductor Alloy Reinforced) y AAAC (All Aluminium Alloy Conductor). Tras el análisis, se seleccionó el conductor Flint, de tipo AAAC, por ser el que mejor se adapta a los requerimientos técnicos del proyecto y a las condiciones climáticas de la región. \\ El departamento del Huila presenta un clima predominantemente cálido, con temperaturas medias superiores a 25 °C y una humedad relativa considerable en varias zonas. Estas condiciones pueden acelerar procesos de corrosión, especialmente en conductores con componentes ferrosos o poco resistentes a ambientes húmedos.
El conductor \textbf{Flint (AAAC)} fue seleccionado por las siguientes razones:

\begin{itemize}
    \item \textbf{Mejor desempeño frente al AAC:} Aunque el AAC tiene excelente conductividad eléctrica, su resistencia mecánica es considerablemente menor, lo que implica estructuras de soporte más robustas y costosas. Además, el AAC es más susceptible a la corrosión en ambientes húmedos, por lo que no es ideal para una zona como el Huila.
    
    \item \textbf{Mayor resistencia mecánica y menor peso:} En comparación con el ACAR, que incluye un núcleo de acero, el AAAC ofrece buena resistencia mecánica sin aumentar el peso significativamente. Esto facilita el diseño de estructuras más ligeras y económicas.
    
    \item \textbf{Resistencia a la corrosión:} Al estar compuesto por una aleación de aluminio resistente, el AAAC ofrece una durabilidad superior en zonas con alta humedad, evitando problemas de oxidación comunes en conductores con núcleo de acero.
    
    \item \textbf{Baja resistencia eléctrica (Rdc):} El conductor Flint presenta una baja resistencia en corriente continua, lo cual reduce las pérdidas por efecto Joule y mejora la eficiencia energética de la línea.
    
    \item \textbf{Compatibilidad con ambientes cálidos:} La aleación del AAAC mantiene su desempeño incluso a temperaturas elevadas, permitiendo una operación confiable en condiciones térmicas exigentes como las del Huila.
\end{itemize}

En resumen, el conductor Flint (AAAC) se eligió por su excelente equilibrio entre eficiencia eléctrica, resistencia mecánica y comportamiento frente a la corrosión. Frente al AAC, que aunque es más económico inicialmente, presenta limitaciones importantes en durabilidad y requerimientos estructurales, y frente al ACAR, que es más pesado y más costoso de instalar, el AAAC se posiciona como la alternativa más adecuada para las condiciones técnicas y ambientales del proyecto en el Huila.