\section{Conclusiones}
\begin{itemize}
    \item Luego de analizar varias configuraciones posibles para la línea de transmisión Huila 230 kV, se eligió una configuración de circuito doble con dos conductores por fase, utilizando el conductor AAAC tipo Flint, con alma Grosbeak de 636 kcmil. Esta alternativa resultó ser la más adecuada porque cumple con todos los requisitos técnicos del proyecto: buena capacidad de corriente, regulación de tensión aceptable, pérdidas dentro del rango esperado y control del efecto corona. Además, al comparar esta opción con otras configuraciones, como el circuito doble tipo tríplex, se evidenció que aunque estas podían cumplir técnicamente, implicaban mayores costos sin mejoras significativas en desempeño. También, la configuración seleccionada se adapta mejor al trazado y facilita la construcción y mantenimiento, haciendo que en conjunto sea más eficiente y económica.

    \item Se analizó que la ruta escogida fue una alternativa ventajosa frente a otras opciones evaluadas, ya que se adapta adecuadamente a la topografía del terreno, presenta buenas condiciones de accesibilidad y minimiza los impactos tanto ambientales como prediales. Esta elección no solo facilita el proceso constructivo, sino que también garantiza el cumplimiento de los objetivos técnicos, económicos y normativos del proyecto. En conjunto, representa una solución eficiente y bien equilibrada para el desarrollo de la línea de transmisión.
\end{itemize}